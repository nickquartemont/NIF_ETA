\chapter{Reproducibility}

\ All of the underlying documentation presented for this research is available in an online repository at \url{https://github.com/nickquartemont/NIF_ETA}. 
Additionally, the codes and documentation are saved internally at the AFIT/ENP department. 
Several Python 2.7 scripts were created to read in data files produced from Sampler as an alternative to the built-in version in SCALE to work more with the data. 
Much of this work may prove useful to others needing the tools created for this work with some simple modifications.
The organization of the repository follows the major efforts taken for the research. A list of tools that will be most beneficial for others is below. The main page includes the thesis, experiment collaboration, documents, briefs, and the models used. 
The main page also includes information on the NIF source and mechanical drawings and pictures of NIF components and ETA. 
Inside of the models tab are the Scale, MCNP, covariance examples, foil information, responses from ORNL, and analyzed outputs. 


\begin{itemize}
	\item Sampler Tools 
\end{itemize}
\begin{sloppypar}	
\url{https://github.com/nickquartemont/NIF_ETA/Models/Scale/ScalePy}
\end{sloppypar}

Instructions for utilizing the tools to read in and analyze response functions from Sampler are described in readme.txt. 
This tool can be directly utilized for response functions text files generated by SCALE Sampler. 
The Sampler files are saved to a pickle file containing the dictionary dataframe of the energy dependent response data. 
	
\begin{itemize}
	\item STAYSL with Sampler Trials  
\end{itemize}

\begin{sloppypar}	
\noindent \url{https://github.com/nickquartemont/NIF_ETA/Models/STAYSL_Unfold/SAMPLER}
\end{sloppypar}	
	
This tool utilizes the Sampler dataframe to generate independent trials for STAYSL and build up the distribution of unfolded responses. 
STAYSL\textunderscore Analysis.py provides the user interface for the tool. 

\begin{itemize}
	\item Fission Product Estimation with GEF  
\end{itemize}

\begin{sloppypar}	
\noindent \url{https://github.com/nickquartemont/NIF_ETA/Models/FissionProduct/GEF}
\end{sloppypar}	

The GEF data has been saved as an Excel file to reduce the size and fit within GitHub's storage restrictions. Users who use the 46 group DPLUS library structure can directly utilize this framework. GEF.py provides the user interface for the GEF data.

\begin{itemize}
	\item Fission Product Estimation with Nagy Fits  
\end{itemize}	

\begin{sloppypar}	
\noindent \url{https://github.com/nickquartemont/NIF_ETA/Models/FissionProduct/NagyFits}
\end{sloppypar}	

The Nagy fit function requires input of the fissioning system or incident energy. Additional isotopes can be added directly to the Excel document containing the experimental data by following the same format. ETA\textunderscore Nagy.py provides the user interface utilized to generate the fission products for ETA. 
	




