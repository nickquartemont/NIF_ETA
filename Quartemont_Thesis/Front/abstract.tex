
\begin{abstract}
% I tried to make this work for an article too. 
% PHYS 598 Format - outline in 5-8 sentences
% 1 - What was studied using what technique and for what purpose (Dense Sentence)
% 2-4 Key result 1,2,3
% 5 - Key conclusion 
% No references/ numbers / equations
% Include a lot of keywords. 
An energy tuning assembly (ETA) was developed to spectrally shape the National Ignition Facility deuterium-tritium fusion neutron source to a notional thermonuclear and prompt fission neutron spectrum to approximate a boosted nuclear device. 
This neutron environment can be used to create realistic synthetic post-detonation weapon debris that contain spectrally accurate fission products across all mass chains to enhance U.S. technical nuclear forensics capabilities for nuclear weapon attribution and device reconstruction. 
This research performed nuclear data covariance analysis through stochastic sampling techniques to predict the performance of the energy tuning assembly to create the objective spectrum, assess anticipated experimental outcomes, and determine the expected fission products to be produced in a highly enriched uranium foil in the sample cavity. 
it was found that the nuclear data covariance affected the neutron fluence energy distribution by a few percent for most of the energy range.
The activation foil activities, used to infer the experimental neutron flux, were found to cover a large range of the neutron energy spectrum but had uncertainties ranging from a few percent to tens of percent due to the nuclear data.
This range of foil activation outcomes was used to show that neutron flux unfolding techniques provided broad spectral agreement between the energy tuning assembly and objective spectrum with an 80+\% probability successful unfolding using the activated foils. 
Additionally, the ETA was also demonstrated as a potential short-pulse neutron source with a 10 shake neutron pulse.
More than 1 billion fissions, approximately of the order collected in nuclear forensics ground samples,  were generated with a cumulative fission product distribution that matched the objective within current predictive capabilities.  
The analysis performed in this research enables the development of the experiment planned for late 2019, enhances confidence in the experimental outcomes, and further develops a unique capability for the technical nuclear forensics community.  


% THis is my first shot before I figured out what the posted notes said. 
%Nuclear weapon environment testing requires a neutron energy spectrum representative of a nuclear event which requires alternative production methods in the absence nuclear weapon testing.  
%An energy tuning assembly was previously designed to spectrally shape the National Ignition Facility laser driven deuterium-tritium fusion neutron source to a notional thermonuclear and prompt fission neutron spectrum to approximate a boosted nuclear device. 
%The energy tuning assembly enabled a representative neutron environment which benefits the technical nuclear forensics community, nuclear weapon certification process, and broad national security applications. 
%A major goal of the energy tuning assembly was to create synthetic weapon debris that contained spectrally accurate fission products across all of the mass chains to enhance nuclear weapon attribution techniques if a nuclear device was used on the United States or allied nations. 
%This work performed an analysis of the energy tuning assembly with a primary focus on the impact of nuclear data covariance through stochastic sampling techniques.
%Simulations in Monte Carlo N-Particle Transport Code 5 and SCALE 6.2 were conducted to predict the performance of the energy tuning assembly to create the objective spectrum and assess the fission products produced in a highly enriched uranium foil in the sample cavity. 
%An activation foil set for determining the neutron flux was incorporated into the experimental energy tuning assembly. 
%The results showed that nuclear data covariance impacted the neutron fluence energy distribution by a few percent for a large energy range of the neutron fluence.
%The uncertainty in the integral results for the activation foils to unfold the neutron flux ranged from a few percent to tens of percent with a large range of neutron energy spectrum coverage.  
%However there was broad spectral agreement between the energy tuning assembly and objective spectrum and high statistical probability of achieving the desired results. 
%The short pulse characterization showed that this may be a potential capability for some neutron induced effects applications. 
%The fission product generation performance produced 2 $*$ 10$^{9}$ $\pm$ 1\% fissions that through experimental data and modeling capabilities had an equivalent cumulative fission product distribution to the objective spectrum.  
%The analysis performed in this research will be compared to the experimental outcomes with the experiment planned for late 2019.  
 
\end{abstract}

