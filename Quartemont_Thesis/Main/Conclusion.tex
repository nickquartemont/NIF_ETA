\section{Modeled ETA Experiment}
\ The objective of the ETA research was to determine if the neutron energy distribution in a ``typical" boosted nuclear weapon detonation can be produced using spectral modification with an energy tuning assembly (ETA) at the National Ignition Facility (NIF). 
This research showed that the ETA concept can fill the technical nuclear forensics and nuclear weapon certification capability gaps that require a spectrally accurate neutron energy spectrum. T 
The correct fission products associated with the thermonuclear plus prompt fission neutron spectrum (TN+PFNS) will follow directly from the neutron flux, which serves as an extremely valuable piece of information for attribution capabilities.
Likewise, an accurate energy distribution of neutrons enhances nuclear weapons certification testing credibility.

\ Since the novel ETA experiment is high cost, understanding the full affect of uncertainties, including nuclear data, is important to capture.
The ETA experiment characterization performed in this research indicates a very strong probability of achieving the surrogate TN+PFNS as designed, but found that the effect of nuclear data uncertainty and covariance on the ETA performance is non-negligible. 
The neutron transport effect on the fluence uncertainty was assessed with the SCALE Sampler sequence with 182 trials and found to be on the order of a few percent over the broad spectrum; however, the systematic uncertainties increase at lower neutron energies.
The statistical testing performed on the ETA-produced neutron fluence compared to the TN+PFNS show large spectral agreement. 
The Pearson correlation coefficient between the nominal results and TN+PFNS was 0.9, which indicates strong agreement between the spectra. 
Also, the Kolmogorov-Smirnov statistic comparing the cumulative distribution functions between the nominal results and TN+PFNS was 0.11 which has a p-value of 0.94 indicating the two samples were drawn from the same distribution with high confidence.  

\ The ETA serves as a candidate for neutron-induced radiation effects testing for nuclear weapon certification. 
The fluence of neutrons in the ETA sample cavity is expected to be 4.9 $\times$ 10$^{11}$ n cm$^{-2}$ $\pm$ 1.4\%, which is near a useful range for radiation effects testing on electronics (10$^{12}$ - 10$^{14}$ n cm$^{-2}$). 
The neutron pulse length for ETA was calculated to be approximately 10 shakes which may be useful depending on experimental timing requirements; however, the combination of fluence, spectrum, and timing provides a unique testing capability that has benefits over alternate U.S. neutron sources.  

\ However, it is worth noting that the current ETA design was not directly optimized to provide a nuclear weapons effects testing capability, and the TN+PFNS is not representative of the transmission neutron flux through material or atmosphere.
Nonetheless, these results provide a step forward toward a short pulse neutron source suitable for the nuclear weapon community. 

\ The foil reaction uncertainties utilized the International Reactor Dosimetry and Fusion File (IRDFF) v.1.05 nuclear data library and were sampled according to a multivariate normal distribution.   
The propagated nuclear data uncertainty on the foil activities result in uncertainties on the order of a few percent for all but the $\mathrm{^{55}}$Mn (n,$\gamma$) reaction where the nuclear data is not as well characterized and the systematic error was found the be 20\%.
The foil activities produced in the ETA cavity are found to be sufficient for gamma-ray spectroscopy post-shot at the NIF.  

Additionally, the activation foil pack designed to unfold the neutron energy spectrum in the ETA experiment is found to have broad neutron energy spectrum coverage and multi-reaction coverage at epithermal energies, typically a trouble area for unfolding. 
The STAYSL unfolded results on each of the 182 Sampler trials provide an 80+\% probability of being able to successfully unfold the neutron spectrum with the foil set and the modeled spectrum based on the $\chi^2$ of each unfolded trial. 

\ In the context of technical nuclear forensics and attribution capabilities associated with device reconstruction, an observable quantity is the fission product distribution created from the neutron flux interaction with the fissile material. 
ETA's modeled performance produces 2 $\times$ 10$^{9}$ $\pm$ 1\% fissions, which is near the order of those collected in forensics ground samples. 
Selected fission products analyzed with the General Description of Fission Observables (GEF) code and experimental data from the literature were used to create energy dependent Nagy fits. 
The fission products produced in the HEU with ETA's spectral shaping capability have an equivalent cumulative fission product yield distribution to the objective TN+PFNS with current predictive capabilities.  
Spectrally accurate fission product distributions are extremely important to nuclear forensics and attribution linked to counter-proliferation efforts and attribution techniques for deterrence. 

\section{Future Work}

The NIF experiment to validate the ETA is planned for late 2019. 
The future work related to the analysis performed here will compare the experimental outcomes to the predicted reactions. 
The experimental results create a verification of the nuclear data covariance analysis technique utilized. 
Updates to this analysis will include changes to the fielded configuration of ETA for the experiment. 
The tools generated for this work will heavily expedite the re-analysis. 

Although ETA is a huge step forward for developing synthetic weapon debris, improvements will be made to develop a second generation ETA.
A THErmonuclear and prompt fission Neutron spectrum energy tuning Assembly (ATHENA) will be developed to generate a more representative neutron spectrum. 
Additionally, facility improvements to the NIF and updated constraints will be incorporated to increase the optimization. 
The goal of ATHENA is to develop a new ETA design to increase the ETA efficiency to produce $\sim 10^{12}$ fissions. 
Attaining a higher number of fissions is important to create better quality samples and achieve better detection of low yield fission products. 
 
Finally, the goals focused on generating a spectrally accurate neutron source and the generation of fission products; however, real-world scenario deposits as nuclear fallout and includes fractionation based on the physical properties and chemistry of the fission products. 
A fractionation technique can most readily focus on refractory fission products with low condensation points, as opposed to volatile mass chains as many of these are gases that may be lost in chemical separations.  
Incorporating the fractionated synthetic fission product debris into a matrix representative to a nuclear forensics collection would be of great benefit to technical nuclear forensics training and exercises. 


