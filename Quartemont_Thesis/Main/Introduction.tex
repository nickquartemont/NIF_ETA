
\section{Motivation}

\ Nuclear deterrence is the cornerstone of U.S. nuclear policy and strategy\cite{Defense2018}. A key component of deterrence theory that enables U.S. strategic objectives is the credibility of the nuclear capability. Two key aspects related to nuclear deterrence credibility are attribution capabilities to hold potential threats accountable and the surety of nuclear weapon systems to function if needed. 

\ The final full-scale U.S. nuclear weapon testing was performed on 23 September, 1992.  The non-proliferation of nuclear weapons and general health and environmental concerns from the radioactive emissions were key drivers for eliminating testing of any kind. The Comprehensive Test-Ban Treaty (CTBT) has banned nuclear explosions for all signatories or supporting nations for an indefinite duration since 1996. A handful of tests have been conducted after the CTBT's effective date; none have been by the U.S.  

\ The 2018 U.S. National Defense Strategy identified the modernization of the nuclear triad as a key requirements for deterrence credibility \cite{NDStrat2018}.
Therefore, there is still a need for the capabilities previously provided through nuclear testing for the study of nuclear environments to support the credibility of the nuclear deterrent.
Previous work has shown that the decision to cease nuclear testing has created a capability gap to reproduce radiation environments of interest to national security applications such as nuclear weapons effects (NWE) and technical nuclear forensics (TNF)  \cite{JointDefenseScienceBoard/ThreatReductionAdvisoryComitteeTaskForce2010, Bevins}. 

\subsection{Nuclear Weapon Certification Capability Gap}
\ Each U.S. administration has supported the requirement and maintenance of a nuclear force structure after the elimination of nuclear tests. President Donald Trump stated at the 2018 State of the Union Address, ``As part of our defense, we must modernize and rebuild our nuclear arsenal, hopefully never having to use it, but making it so strong and powerful that it will deter any acts of aggression" \cite{Trump2018}. The National Nuclear Security Administration (NNSA) is tasked with the mission of maintaining the nuclear stockpile's safety, security, and effectiveness under the Stockpile Stewardship Program (SSP). 

\ Full scale system testing in relevant environments is generally recognized as a critical requirement for nuclear weapon certification, just as it is for any Department of Defense (DOD) weapon system. Actual system tests cannot be performed, so demonstration of components or subsystems in a relevant environment is an important part of the technology readiness level as part of the DOD Instruction 5000.02 series \cite{DODI50002} and the DOD nuclear certification process specified in DOD Directive 3150.02 \cite{DODD315002}. 
Representative nuclear weapons system and effects testing supporting the SSP is carried out by the Department of Energy (DOE), DOD, national laboratories, and supporting organizations. The scope of the testing sites is incredibly wide, ranging from radio frequency communications to the prompt gamma and neutron emissions following a nuclear event. %A summary of some of the nuclear weapons effects testing simulation and facility capabilities is shown in Table \ref{tab:NWECap}. 
Some testing is conducted on components of the nuclear weapons themselves, such as the near-system-level hydrodynamic tests performed with inert pits \cite{martz2014without}. However, many aspects of nuclear weapons are only testable via computational methods or experiments which may not truly represent the physics involved in a nuclear weapon. 
Not employing full scale nuclear testing accentuates some uncertainty in nuclear force credibility, so  alternative testing methods are of extreme importance to the nuclear force structure. 

\ One important gap identified is the availability of neutron environments for testing at current U.S. facilities in comparison with the environment that a nuclear weapon would experience or produce \cite{JointDefenseScienceBoard/ThreatReductionAdvisoryComitteeTaskForce2010}. 
Current U.S. neutron sources do not have an accurate energy or temporal distribution for the nuclear environment that nuclear systems are required to survive in certification testing. 
This problem is complicated further as the transmitted neutron flux through the physical environment and to the target varies significantly in energy and temporal distribution depending on the scenario and system being considered.  
Furthermore, the neutron fluence and energy spectrum internal to the weapon cannot be directly measured but must be inferred from sources such as activation products. 
The lack of a relevant facility has led to a reliance on simulations and large engineering safety factors\cite{JointDefenseScienceBoard/ThreatReductionAdvisoryComitteeTaskForce2005}.  To address this capability gap, it would be beneficial to have a neutron environment testing capability with an accurate neutron energy and temporal profile.

 
\subsection{Technical Nuclear Forensics Capability Gap}
\ A key strategy for countering nuclear terrorism identified in the 2018 Nuclear Posture Review is the importance of ``deterring state support for nuclear terrorism through advanced forensics and attribution capabilities"\cite{Defense2018}. To this end, the technical nuclear forensics (TNF) community requires the ability to generate representative post-detonation debris samples for training and development of attribution techniques.  The generation of accurate fission product inventories in the representative debris is both extremely important for the attribution and very difficult to do with existing U.S. facilities due to a diminishing pool of subject matter experts and outdated facilities \cite{NAP12966}. Additionally,  fission debris is paramount to the nuclear device reconstruction capabilities \cite{Fedchenko2015}. 

% Suggest that you add a para here expanding this thought light you did above with radiation testing.  You might consider building it around this quote from the 2013 report from the Joint Nuclear Forensics Working Group:
According to the Joint Nuclear Forensics Working Group report from 2013, 
\begin{quote}
	Current post-detonation debris analysis techniques derive largely from the nuclear weapons test programs of the Cold War. Leveraging the Cold War infrastructure enabled a baseline forensics capability to be established quickly, but has resulted in a capability that relies largely on science and technology developed in the nuclear-testing era, with timelines and priorities sometimes distinct from those of nuclear forensics. In addition, current analysis methods are often labor-intensive, and rely on education and training that are no longer prominent in the U.S.\ university system \cite{JNFWG}. 
\end{quote}
\ Advances in attribution capabilities for TNF require facilities that produce nuclear weapon relevant environments which drives the distribution of observed fission products. The attribution problem is also complex in that chemical and physical processes post-detonation can drastically impact the debris.  The generation of realistic synthetic weapons debris would be of enormous benefit to the TNF community for training and research to improve the nation's forensic-based attribution capabilities.

\ A primary component of the debris critical for these capabilities is the fission product inventory in the debris. 
Post-detonation fission product analysis provides a means of determining many characteristics of a nuclear device. 
In particular, according to a U.S. National Research Council report from 2009, the fission debris can provide the most accurate measurement of weapon yield when combined with device information \cite{USNRC1}. 
Additionally, the CTBT utilizes fission products to verify compliance with the nuclear test ban \cite{Fedchenko2015}. 
% These seem random to me
%Numerous fission products are of great importance for varying aspects of nuclear sciences. A couple notable examples are  $\mathrm{^{90}Sr}$ and $\mathrm{^{14}C}$ which are used for estimating dosage received from past nuclear weapons testing \cite{Radiation2000}.

\subsection{Neutron Environment Capability Gaps}
The capability gaps outlined for nuclear weapons certification and TNF motivate the need to generate spectrally accurate nuclear weapon neutron environments.  
In particular, the present U.S. testing capability does not have the ability to produce neutron spectra that combine a thermonuclear (TN) and prompt fission neutron spectrum (PFNS).  
The vast majority of U.S. testing facilities are focused on the Watt-fission spectrum, while a few are capable of producing the 14.1 MeV TN component from the deuterium-tritium (DT) fusion process\cite{Bridgman}. 
Several examples of U.S. testing facilities for prompt neutrons outlined in Figure \ref{fig:CompSource} are the Sandia Pulsed Reactor III (SPR), Sandia Annual Core Research Reactor (ACCR), White Sands Missile Range (WSMR) Fast Burst Reactor (FBR), the Los Alamos National Laboratory (LANL) Rotating Target Neutron Source (RTNS), and the LANL Weapons Neutron Research facility (WNR). 
The differential spectral profile of these sources compared to a notional TN+PFNS is shown in Figure \ref{fig:CompSource}. 

\begin{figure}[ht]
	\centering
	\includegraphics[width=\linewidth]{Figures/Chapter1/SourceComparison.png}
	\caption[Comparison of selected neutron sources to notional TN+PFNS.]{Comparison of selected neutron sources to notional TN+PFNS {\cite{Bevins}}.}
    \label{fig:CompSource}
\end{figure}

\ Each of the available neutron sources has an important purpose for national security applications; however, they cannot meet the energy and temporal spectrum for every nuclear testing requirement. 
In comparison with the TN+PFNS, nearly all of the neutron sources are heavily weighted to lower energies and do not contain enough high-energy neutrons to represent the TN component of a nuclear weapon. 
The RTNS has a high-energy component, but the magnitude of the flux is substantially lower than required for nuclear hardness applications where the timing profile and integral fluence is important.  
Additionally, these large facilities are often at risk of shutdown, such as the SPR-III decommissioning for storage at the Nevada Test Site in late 2006\cite{SandiaNationalLaboratory2007}. 
Others, such as WSMR FBR, are discussed for shutdown with growing regulatory demands and security requirements for storing highly enriched uranium (HEU) \cite{UnitedStatesNuclearRegulatoryCommission2018}.
Gathering accurate experimental results requires a neutron flux spectrum equivalent to that of a true nuclear event, which creates a need for a neutron source capable of emulating the environment. 
Therefore, development of a TN+PFNS source would enable production of the correct fission product inventory in surrogate debris and thereby enhance the ability of the TNF community to perform the attribution mission. 
Additionally, a TN+PFNS source capable of NWE testing would greatly improve the nuclear weapon certification process. 

\section{Background}

\ Many approaches can be used to create nuclear weapon-relevant neutron spectra in the absence of full-scale nuclear weapons testing. Some mechanisms are only applicable within different communities in the nuclear sciences. Four main ways that the neutron environments can be approximated for synthetic fission product debris production are sample doping, direct production using fission converters, surrogate methods, and spectral modification of existing sources\cite{Bevins}. In the context of neutron effects on electronics, the key approaches utilize existing sources, computational models, and surrogate charged particle reactions\cite{Bevins, Bouchard}. Each of these methods are limited in representing the neutron environment experienced in a nuclear weapon. 

\ The sample doping technique is accomplished by selectively correcting mass chains to modeled equivalent ratios. The resultant sample is built so as to look like it was produced with a desired energy dependent fluence. 
A TNF application using sample doping is the production of glass surrogate fallout debris for use in exercises or training\cite{Carney2014b}. 
The glassy matrix is created to emulate the solidified fission debris and entrained environment that is swept up in the stem of a nuclear explosion. 
The glass is doped with uranium and irradiated under various neutron environments depending on the requirements; however, the irradiation is often done with a thermal neutron reactor. 

\ A key deficiency with utilizing a thermal reactor is that the neutron energy spectrum is not a close approximation to a weapon spectrum, and the resultant fission product ratios that follow will therefore not be accurate either. Utilizing a harder, or higher energy, neutron spectrum reactor is a better approximation; however, it is still not an accurate representation of the fission product distribution. The valley fission products will be significantly lower than for a higher energy weapon spectrum.

\ Additionally, the sample doping technique can be approached by irradiating different samples at different facilities. A final sample which has the ``correct'' fission product ratios can be created by selectively pulling mass chains from  the irradiated samples. 
This sample doping technique creates a useful fission product debris sample; however, the spectral and temporal nature of the sample is not equivalent to what would be produced in a real nuclear explosion. 

\ Direct production using fission converters utilizes nuclear reactions to create a shaped neutron flux, which can be done via charged particle interactions or through fusion sources with a fission converter \cite{edsstc.32099219990101}. 
It has been shown that direct production is ``impractical, complex, and unlikely to be implemented for safety or technological limitations"\cite{Bevins}. 

\ Surrogate methods rely on the formation of an equivalent compound nucleus through an alternative reaction mechanism\cite{DIETRICH2007237,Scielzo2012}. 
Surrogate methods are popular in studies where forming the product nucleus through the desired reaction is difficult or the energy cannot be fine-tuned. 
An example of this is neutron induced fission on $\mathrm{^{235}}$U where a possible surrogate for $\mathrm{^{235}}$U neutron induced fission reaction, (n,f), is $\mathrm{^{232}}$Th ($\alpha$,f), both of which form the $\mathrm{^{236}}$U compound nucleus. 
The surrogate approach has seen success; however, the nuclear data supporting the reactions is not as well understood\cite{RevModPhys.84.353,Narek1}. 
Additionally, there are some assumptions on the compound nuclear equilibration and spin-parity state which can impact the decay channels of the studied reactions \cite{DIETRICH2007237}.

\ Another commonly used alternative reaction surrogate method is to utilize charged particles for neutron damage in radiation effects on electronics. 
Ion beams can be used as a surrogate for neutrons by comparing the relative displacements per atom caused by the charged particle compared to a neutron\cite{Galy2018}. 
A major benefit of using ion beams is that the energy can be finely tuned both in energy and deposition location, whereas neutrons are not as easily controlled. 
A disadvantage of using charged particles is that a large portion of the energy deposition as it travels through materials is based on electronic stopping power, while the neutral neutrons have negligible electronic interactions. 
Neutrons have larger mean free paths in materials than larger charged particles of the same energy, so the flux will also be different. 

\ The Qualification Alternatives to the Sandia Pulsed Reactor (QASPR) program is the most significant venture into the use of surrogate ions to perform neutron effects component level testing as a replacement alternative for the SPR\cite{JointDefenseScienceBoard/ThreatReductionAdvisoryComitteeTaskForce2010}. 
QASPR combines operational irradiation facilities with modeling to predict neutron effects on electronic performance. 
While there have been substantial improvements to \mbox{increasing} the verification and validation of simulated charged particles to experimental outcomes, the validation for the experimental data benchmarked to neutron experimental data is lacking in many cases\cite{Bouchard}.

\ The last main approach to create accurate energy distribution neutron environments that could be used is spectral modification,  a method of altering a neutron spectrum through nuclear interactions to generate an energy spectrum of interest. 
Fundamentally, spectral modification is the goal of moderated nuclear reactors to increase efficiency and allow the use of low enriched uranium fuel. 

\ Spectral modification is also performed in beam shaping assemblies used for boron neutron capture therapy (BNCT), where neutrons are used to treat tumors through neutron capture reactions in boron. 
An optimized objective neutron spectrum focused on the epithermal region is published by the International Atomic Energy Agency (IAEA) \cite{Kasesaz2016}. 
BNCT has been explored with a wide variety of sources including accelerators and deuterium-deuterium (DD) fusion. 
A beam shaping assembly can be designed to moderate a source neutron flux to appropriate thermal, epithermal, and fast spectra for BNCT\cite{Ardana2017}. 
The build up of a design is produced primarily through moderation, reflection, and collimation of neutrons to the patient \cite{Zaidi2018}. 
However, the approach to designing a beam shaping assembly lends itself to inefficiencies from an energy and population perspective.  The collimation process blocks out a portion of potentially usable particles. Additionally, the beam shaping assembly resultant spectrum is often under-optimized. The development process could be enhanced to increase efficiency and spectral profile agreement with the objectives. 

A novel spectral modification approach was developed by the University of California-Berkeley and Lawrence Livermore National Laboratory (LLNL) for the development of an energy tuning assembly (ETA) to modify the National Ignition Facility (NIF) source to produce a TN+PFNS \cite{Bevins}. 
To perform the spectral modification, the Coeus metaheuristic optimization software package was developed to avoid manpower intensive iterative studies and enable the rapid design of future ETAs to convert a facility's characteristic source spectrum to any arbitrary objective spectrum, within the constraints of physics\cite{Coeus}. 
%For their study, the design constraints included the mechanical envelope at the NIF, a 75 kg mass limit, material limitations, and efficiency requirements. 
%The nearly-globally optimum solution produced by Coeus was modified slightly to reduce cost and improve manufacturability. 
% I'd eliminate
Gnowee, the Coeus opimization engine, was developed for ``rapid convergence to nearly globally optimum solutions" of this class of engineering problems\cite{Bevins2018}. 
It is important to note that the Gnowee and Coeus codes have applicability over a wide range of engineering problems, not just for the production of a TN+PFNS source.  

The result of the ETA design produced an acceptable representation of the TN+PFNS with the associated fission product distribution. 
The ETA design has been built and preliminary validation tests were conducted at the Lawrence Berkeley National Laboratory’s 88-Inch Cyclotron \cite{Bevins, Stickney}. 
The preliminary validation utilized 33 MeV deuterium breakup on tantalum as a neutron source and investigated the ability to model the ETA performance \cite{Stickney}.
Integral validation is planned in fiscal year (FY) 2019, and a development shot to enhance ETA performance is planned in FY2020. 

 
\section{Problem} \label{problem}

There are several deficiencies in the previous work that need to be addressed \cite{Bevins}. The broad research objective for ETA is \textit{Can an accurate neutron energy distribution expected from a ``typical" thermonuclear or boosted nuclear weapon detonation be produced using spectral modification at the NIF}?
This research effort aims to address three main problem areas for the fiscal year (FY) 2019 ETA experiment and spectral shaping of neutron sources for simulating nuclear weapon environments that were raised by previous work by incorporating nuclear data covariance analysis.
A modeling component that also needs to be characterized is utilizing a full scale NIF model to determine the entire contribution to the neutron flux.  
Additionally, ETA needs to be characterized as a potential `short pulse' neutron source (SPNS). Each are detailed below along with accompanying research objectives.  

\begin{enumerate}
	\item FY 2019 NIF shot (ETA): Systematic uncertainty was not fully addressed in the previous ETA calculations
	\begin{itemize}
		\item Quantify the impact of nuclear data covariance on the simulated results for the neutron energy spectrum, foil activation rates, and fission product production rates
		\item Design a foil activation diagnostic pack to provide increased resolution in the epithermal neutron energy range
		\item Prioritize and estimate production of fission products for radio-chemical analysis using recently published data
	\end{itemize} 

	\item The ETA at NIF was not previously evaluated for use as a potential SPNS
		\begin{itemize}
			\item Model the neutron timing profile and expected flux in the ETA experimental cavity
		\end{itemize}		

%	\item FY 2020 NIF shot (ATHENA): The current ETA efficiency is too low for use as a production NTF and NWE source %you should introduce the NWE term somewhere above
%		\begin{itemize}
%			\item Develop a more representative neutron spectrum
%			\item Update facility constraints to reflect recent NIF upgrades %TANDEM
%			\item Develop a new ETA design to increase the ETA efficiency to produce $\sim 10^{12}$ fissions
%		\end{itemize}		 
	
%	\item FY 2020 NIF shot (ATHENA): The ETA at NIF was not evaluated for use as a `short pulse' neutron source (SPNS)
%		\begin{itemize}
%			\item Model the neutron timing profile and expected flux
%			\item Incorporate the ability to measure the neutron time profile into the FY 2020 ETA design
%		\end{itemize}		

\end{enumerate} 


\section{Questions and Hypothesis}
The research questions and hypotheses associated with the problems outlined in Section \ref{problem} are detailed below.  
They are organized by the problem and capability that they support.

\begin{enumerate}
	\item 2019 ETA Fission Product Experiment
	
	\begin{itemize}
		\item \textbf{What is the effect of nuclear data covariance on the simulated results?} It is expected that including nuclear data uncertainty will increase the relative error by approximately 1\% for integrated and well understood reactions and may extend over 10\% for less studied reactions thereby dominating Monte Carlo statistical uncertainty. 
		
		\item \textbf{Does the activation foil pack have sufficient coverage of the neutron spectrum to be used for unfolding?} Previous work indicated that the current foil pack design has poor coverage in the epithermal region and is not sufficient to robustly unfold the neutron spectrum should the model deviate from experimental results \cite{BEVINS2019}.
		 % insert citation once accepted; I'll send the draft. I think Stickney's thesis is a good placeholder citation. From what I can tell. NIMA ETA still isnt out there. 
		 Incorporation of better foil characteristics will improve this deficiency, and the performance can be tested through unfolding the ETA generated neutron spectrum using perturbed samples generated from including the nuclear data uncertainty.
		 
 		\item \textbf{Does the simulated ETA fission product distribution agree with the expected TN+PFNS distribution?} It is anticipated that the fission product distribution produced with HEU in ETA's sample cavity will match the TN+PFNS fission product distribution, and  previous work has shown agreement between the two \cite{Bevins2018}.
		
%	\end{itemize}
%    \item 2020 ATHENA Surrogate Debris Experiment %I think we can really push towards this by then
%    \begin{itemize}
%    	\item \textbf{Can the ETA efficiency be increased to achieve $\mathbf{10^{11} - 10^{12}}$ fissions?} This will be a factor of 100 to 1000 over the original ETA design.  The gain will be possible due to NIF source development, changes to the design envelop, updated TN+PFNS objective spectrum, and changes to the optimization objectives. 
    	
%    	\item \textbf{How well does the enhanced ETA perform to match the objective neutron environment?} The chi-square statistic will be used similarly to the goodness of fit criteria for the original ETA design. % The optimization used a flux weighted relative least squares fit - are you planning to use the chi sqaured? Also, there is not a hypothesis here...I'd probably delete? 
    \end{itemize}
	\item ETA SPNS Characterization 
	\begin{itemize}
		\item \textbf{Can an ETA be useful as a capability for testing of prompt neutron environments?} 
		%The original ETA design has a neutron pulse length on the foils of approximately 1 microsecond. 
		It is anticipated that ETA can provide a TN+PFNS electronic testing capability due to the short NIF neutron pulse ($\sim$ 300 ps), although the sample cavity is smaller than would be required for larger component testing. 
	\end{itemize}
\end{enumerate}

\section{Assumptions and Limitations}

An omnipresent limitation in many studies of science and engineering is the quality and quantity of available data for applications. 
Nuclear engineering commonly draws from published works containing the relevant nuclear data and the uncertainties behind them. 
There is also uncertainty in the published uncertainties as much of the available data is derived from models and never directly tested. 
The results presented in this document are limited by the currently accepted understanding of nuclear physics phenomena and by the limitations of published data that are consistently being improved upon by the nuclear science community. 

The second assumption of this work is that the nuclear covariance follows a multivariate normal distribution. Further analysis of this assumption is outlined in Section \ref{Multi}. 
Additionally, the uncertainty was assumed to be relatively insensitive to group structure. 

One delimitation, which is done so for convenience and publishing ability, is that the nuclear weapon environments are presented at an unclassified level. 
All information used to develop the neutron flux and profile is available in open literature or derived from unclassified information to produce a representative environment. 
The accuracy of the representative neutron environment compared to a specific real-world nuclear weapon scenario was not analyzed and will not be presented. 
The scope of this work aims to provide a position where, if desired, one could easily go from the unclassified spectrum to one that fully meets a requirement. 

An assumption for this work is the NIF is the most effective choice of the neutron source, and that the NIF will be operational. 
Other sources may exist that would also perform the role, but NIF has unique benefits such as the prompt nature of the neutron yield and the fast neutrons arising from DT fusion. 
Although the NIF has been in operation since approximately 2010, there is a potential insertion of systematic error based on the source characterization and variability in the source output. However, any changes to the magnitude of the NIF source output will produce linear responses to the results shown here, so determining the source strength is not a high risk item.  
Additionally, the NIF geometric uncertainty is considered negligible due to rigid tolerances for the positioning systems.  

The TN+PFNS as an objective spectrum was assumed for this work. 
Nuclear weapons can be categorized into three general classes: fission, boosted and TN\cite{Bridgman,Mctl}. 
Research has shown that the majority of the present capability to produce synthetic debris is focused on fission devices \cite{Bevins}. 
The TN+PFNS was chosen because it is an area that lacks substantial source development \cite{Bevins}. 
It is important to note that there is not just one spectrum that can classify the TN+PFNS. 
The TN portion of the weapon spectrum is assumed to be pure DT fusion \cite{Mctl}. 
The impact of weapon design, which can vary substantially and play a large role in the resultant neutron energy spectrum, is not evaluated in this work. 

Some physical phenomena present in a full scale nuclear event are not taken into consideration for this analysis. 
First, the temperatures achieved in nuclear weapons are on the order of $10^{7}$ K, which is not experimentally feasible for configuration into the NIF \cite{Glasstone1977}. 
Second, the time dependency of the internal neutron flux as the weapon is configured is not taken into account. 
Additionally, there will be large changes to the flux from initiation to burnout;  this work only considers a time and volume average result. 
Third, the synthetic weapon debris is created without induced fractionation.  In a real nuclear detonation chemical fractionation occurs when the nuclear debris formed solidifies based on the condensation point of the constituent materials. 
Finally, the neutron spectrum considered is the internal weapon spectrum which would be attenuated in magnitude and energy through material and the atmosphere. 
For fission product generation, the internal weapon spectrum is the key item of interest; however, nuclear certification testing would require a modified objective spectrum. 

\section{Approach}

The spectral shaping problem was defined by the objectives and constraints. 
For this research, the objectives for ETA were the TN+PFNS and the ultimate generation of spectrally accurate fission products.
The problem constraints were based on the NIF source term and mechanical envelope. 
The input objectives and constraints were utilized in Coeus to produce a nearly-globally optimum solution for an ETA\cite{Bevins}. The constraints for the problem were governed by the NIF polar direct drive exploding pusher (PDXP) source, stay-out angle defined by the incident lasers to drive the fusion, and the constraints of the NIF Target and Diagnostic Manipulator (TANDM).
The work performed previously completed a baseline design for the original ETA that will be used for  analysis of the expected experimental performance \cite{Bevins}. 

%Coeus is used as an optimization tool to develop ATHENA\cite{Bevins2018}.  The objective spectrum for ATHENA is an improved TN+PFNS with the goals of increasing the number of fissions in the HEU sample and better representing the neutron environment. The ATHENA objective spectrum is also appropriate for neutron effects on electronics testing for high altitude burst or space environments. The neutron energy distribution is modified by the atmosphere; however, negligible attenuation occurs in a near vacuum.  

\ The point design was modeled with MCNP5, MCNP6, and SCALE version 6.2 to perform neutron radiation transport \cite{MCNP5, MCNP6, SCALE}.
MCNP was used for the continuous energy solution, while SCALE was used for group-wise nuclear data covariance analysis. Additional post-processing incorporated nuclear data uncertainty associated with the activation cross-sections. 
MCNP versions 5 and 6 were both used depending on compatibility with the surface source read (SSR) files generated by LLNL for a full NIF model simulation to account for ``room return" and scattering off ancillary equipment.
Utilizing two different radiation transport models also increased the degree of confidence in the results. 
The radiation transport simulations provided results for the reaction rates for foil activation, neutron energy spectra, and temporal aspect of the neutron flux. 

\ The General Description of Fission Observables (GEF) code was utilized for developing the expected fission product yields \cite{Schmidt2015}. 
GEF is a Monte Carlo and theory based approach that incorporates experimental data to determine fission observables, such as fission product yields \cite{Schmidt2016}. 
Empirical methods for determining fission product distributions also exist as alternative approaches to GEF. 
A formulation of this fit by S. Nagy was also used and is beneficial for comparison to GEF in addition to providing isotope yields \cite{Nagy1978}. 
These empirical methods often include simplifications, such as ignoring neutron multiplicity, to create a simpler equation and more direct tie to existing data \textemdash both a benefit and limitation of this approach.  

\ A foil pack designed to be placed in the ETA experimental cavity was created to successfully unfold the incident neutron spectra from the activation foils. 
The activation foils were selected with many important factors including the confidence in the nuclear data and energy range covered. 
The modeled foil activities were used with the underlying nuclear data to unfold the neutron spectrum using Pacific Northwest National Laboratory (PNNL) STAYSL. 
STAYSL relies on least-squares spectral adjustment based on the chi-squared of the measured activities to determine the incident neutron flux \cite{Greenwood2016}. 

\section{Innovations}

This research advanced the field of nuclear science and engineering in a few key ways:

\begin{enumerate} 
\item \textbf{Demonstrated further abilities to incorporate nuclear data covariance into radiation transport simulations:}  
The standard methodology for determining nuclear data uncertainty from stochastic sampling approaches is discussed in Chapters \ref{chap:Literature Review} and \ref{chap:Methodology}. This work utilized an approach to encompass the full range of uncertainty in nuclear reactions when sampling from a multivariate normal distribution thereby generating a more accurate depiction of the resultant uncertainty. 
\item \textbf{Improved the methodology to generate synthetic fission product debris:}
A major goal of this research is to provide an improvement in spectrally accurate fission product debris production and improve the ability to model the production and predict the resulting debris. 
\item \textbf{Advanced the field of neutron spectral shaping:}
The ETA design characterization represents a stepping stone in nuclear certification testing for providing a time- and energy-representative neutron environment.
\item \textbf{Developed methodology for quantifying the neutron flux uncertainty for foil activation unfolding of neutron energy spectrum:}
The techniques to map the systematic nuclear data uncertainty to an arbitrary group structure are discussed in Chapter \ref{chap:Methodology}. 
\item \textbf{Contributed to future improvements of SCALE:}
Feedback was provided to Oak Ridge National Laboratory (ORNL) for future improvements to the SCALE package including inconsistent uncertainties from published data, the need for parallelization in individual Monte Carlo simulations, and the need for a high energy group structure with covariance data.  
\end{enumerate}

