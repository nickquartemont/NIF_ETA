\documentclass[10pt,journal]{ieeetran}
%  Replace the top line with these two lines for the Optics Letters Template
%\documentclass[10pt]{article}
%\usepackage{osajnl2}

\title{LaTeX: Quick Start Guide for the AFIT Network}
\author{Lt Col M. Hawks}
\renewcommand{\footnoterule}{\rule{3.5in}{0.25pt}\vspace{4pt}}

\begin{document}

\maketitle

\begin{abstract}
There are numerous guides to LaTeX commands, but knowing commands
is no help unless you also know how to set up and use a LaTeX
compiler to turn those commands into a document.  There are also
guides on-line that tell you how to download and install
everything you need, but since none of us have admin privileges
that's not very helpful.  Fortunately, (almost) everything you
need is already on the AFIT network.  So forget those other guides
(for now)...here is how to get started with the tools available on
the AFIT network.
\end{abstract}

\section{Why Latex?}
First, you might be thinking this sounds like a pain, so why
bother? The reason is pretty simple: LaTeX does what you tell it
to do. There is some pain up front while you learn how to use it.
But you can be sure Word will make you pay too---by randomly
moving figures, incorrectly numbering equations, and so on. Pick
your poison. For $small$ documents, Word is fine. For a large
technical document with lots of equations, figures and references,
it is (in my opinion) a real beast.

\section{Software Setup}
There are a lot of ways you could do this, and lots of sources
that describe in general terms what to do.  Most of the advice you
can get on-line, however, doesn't really work here because you
cannot install software on the network.  What you need to know is
how to use what SC has already installed.

There are (again) many solutions, but the simplest way I know of
to get up and running is to just run
\begin{ttfamily}R:$\backslash$LaTeX$\backslash$WinEdt5.5$\backslash$WinEdt.exe\end{ttfamily} and
let WinEdt do the hard stuff for you.  WinEdt is basically just a
text editor, but it is linked in to MiKTeX (a Latex compiler) and
some other convenient apps that makes it very simple to use.

\section{Document Files}
There are really two things you need here.  A .cls
file\footnote{You can get away without a .cls file by using some
built-in classes.  For publications and theses you'll always need
the .cls file, so lets just get used to it now, huh?} which
defines your template and a .tex file which contains all your
content.  Making a template file is \emph{hard}, but there are
many available on-line ... just use an existing one!  In fact,
taking an existing .tex file (like this one) and modifying it
isn't a bad plan either.

Just put the .cls and .tex file in the same folder (a \emph{local}
folder please, not in
\begin{ttfamily}R:$\backslash$LaTeX$\backslash$WinEdt5.5)\end{ttfamily}. You
should also make sure the document class (near the top of the .tex
file) matches your template name. For this file, I'm using the
ieeetran.cls template, so the first line of my .tex file looks
like
\begin{verbatim}\documentclass[10pt,journal]{IEEEtran}.\end{verbatim}
The stuff in square brackets just chooses some options defined in
the template.

You don't really need to open the .cls file or do anything with
it.  Just put it in the folder and press on.

\section{Viewing your Document}
LaTeX is basically a programming language, and your document is
source code.  It needs to be compiled just like any other code. To
convert your source code into a document,
\begin{enumerate}
    \item save your file.
    \item click the Texify button (little wolf(dog?) icon) in
        WinEdt. This creates a preview, which is stored in a
        .dvi file.
    \item to see the preview, click on the DVI Preview button
        (magnifying glass).
    \item if you're happy with the preview and want to make a
        'real' document out of it, click the
        dvi$\rightarrow$pdf button.
\end{enumerate}
There are a lot of other ways to get from code to document, but
this works to get you started.  You can play around with different
combinations.

\section{LaTeX commands}
Now that you can create a document, you probably want to know what
to put in it.  There are a $lot$ of sources for that.  There are
several guides available in \begin{ttfamily}L:$\backslash$enp
students$\backslash$LaTeX\end{ttfamily}\footnote{of particular
interest are \emph{LaTeX Quick Reference v1-1.pdf} written by
AFIT's own Lt Col Anderson, and the new Thesis Style guide by Lt
Col Anderson and Maj Magnus.}. There are also many more resources
on-line. If you get serious about it, there are also several
books; I like \emph{Guide to Latex} by Kopka and Daly (about \$50
on Amazon), but that's just me.

Some additional sources:
\begin{itemize}
    \item  http:\/\/www.ctan.org\/
    \item  http:\/\/www.tug.org\/
    \item  http:\/\/stommel.tamu.edu\/~baum\/tex\/tex.html
    \item
        http:\/\/www.nhn.ou.edu\/~morrison\/LaTeX\/index.shtml
\end{itemize}



\end{document}
