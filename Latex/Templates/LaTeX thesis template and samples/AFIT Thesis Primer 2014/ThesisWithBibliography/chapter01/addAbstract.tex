
\noindent The next section to add to the front matter is an abstract.
Create a file {\em abstract.tex} and place the text for the abstract
between the commands {\bf $\backslash$begin\{abstract\}} and {\bf
$\backslash$end\{abstract\}} as below.

\vspace{-0.3in}
{\singlespace
\begin{verbatim}
\begin{abstract}
    Midwave Infrared Imaging Fourier Transform Spectrometry analysis of
    plume data lends itself to an understanding of the combustion
    chemistry involved with the source. ...
\end{abstract}
\end{verbatim}
}

Above, we use a construct called an environment.  There are several
enviroments: figure, itemize,
verbatim, quote, equation to name a few.  \Latex friendly editors will
help you build these environments.  The abstract environment is
actually a customized environment created in the {\em afitThesis.sty}
file; thus, it will not be found in the common \Latex literature or
tools; but, as you can see above, it is simple to implement.


