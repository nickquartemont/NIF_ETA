\begin{abstract}

This primer aids the AFIT student in generating the first draft of
their thesis using \Latex. The primer is produced according the tenets
described within the document.  All source code is provided in a zip
file posted to \primerAddress.  The file structure of this zip file
demonstrates a practical way to organize a thesis with its supporting
materials and---further---illustrates how your document can be produced
with version control.

\end{abstract}
    

%     As an AFIT graduate student, you are about to write one of the
%     longest documents of your career.  Whether you use a ``what you see
%     is what you get'' (WYSIWYG) interface like Microsoft Word
%     \trademark or a typesetting system like \LaTeX\ldots when you
%     create a digital document, you are writing a program.  The larger
%     any program is, the more bugs it will contain and the more likely
%     the program will result in a catastrophic error.  Common bugs
%     result in improper formating, double words, lost sentences,
%     and---in the worst case---corrupted, unreadable files.
% 
%     Typesetting systems such as \LaTeX\ limit the new lines of code
%     generated with each new document.  As a result, they produce
%     cleaner, easier-to-debug products.  Additional benefits of \LaTeX\
%     are rapid reformatting and high quality typesetting of equations
%     and vector graphics.
