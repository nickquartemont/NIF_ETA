To compile a LaTex document, start simple with the code listed 
below.  Store the code as a .tex file in your
Thesis directory. Then compile the code to test the set up of your \Latex 
distribution and compiler\footnote{Popular compilers include TeXworks 
and TeXShop.}.  Figure~\ref{fig:MyFirstLaTeX} provides a screen
shot of the typeset document with its compilation aids.

 {\singlespace   
 \begin{verbatim}
\documentclass[12pt,letterpaper,oneside]{book}
 
\begin{document}
The quick brown fox jumped over the lazy dog.
\end{document}
\end{verbatim}
} 
\noindent The code has two parts: the preamble and the body.  The
preamble establishes the default formatting for the document; the body
holds the content.  The preamble starts with a {\bf $\backslash$documentclass}
declaration and ends at the {\bf $\backslash$begin\{document\}}
command. The {body} is placed in between the {\bf
$\backslash$begin\{document\}} and {\bf $\backslash$end\{document\}}
commands. 

In the preamble of this first document, Here we have selected a one
sided, 12-pt font book format.  In the body, let us enter a short
phrase---just to get a feel for how content is added---that includes 
all characters in the Roman alphabet.
    
   
%Other formats include article, report, memoir The body contains the
%actual content.


