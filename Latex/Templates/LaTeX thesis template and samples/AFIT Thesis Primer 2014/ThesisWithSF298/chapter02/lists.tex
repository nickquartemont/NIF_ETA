\Latex provides macros that automatically generate lists.  These lists
are the table of contents, the list of figures, and list of tables;
they are generally placed in the front of the document.  The afitThesis
style file defines two more lists: a list of abbreviations and a list
of symbols.  These lists are not required in an AFIT publication but
may prove useful.  In this section, we show how to implement these
lists in your document.

\subsection{Creating a list of abbreviations}

Given the diversity of acronyms in defense publications, it may
be wise to add a glossary that defines the abbreviations used in a
document.  Use the following commands to implement a list of
abbreviations:

\begin{description}
    \item[$\backslash${listofabbreviations}] ~\\Produces a list of
    abbreviations with entries from all the $\backslash${abbreviation}
    and $\backslash${abbreviationFull} commands in the body of the
    document.

    \item[$\backslash${abbreviation}[{\em definition}{]}\{{\em
    acronym}\}] ~\\Adds {\em acronym} to text and the {\em
    acronym} and optional {\em definition} to the list of
    abbreviations.

    \item[\bf $\backslash${abbreviationFull}[{\em definition}{]}\{{\em
    acronym}\}] ~\\Use $\backslash${abbreviationFull} as an alternate
    to $\backslash$abbreviation when you wish to place the {\em
    definition} followed by its {\em acronym} in parentheses in the
    text.
\end{description}

\noindent To implement the abbreviation commands within your text, the code
below\ldots

\begin{verbatim}
Here is an example of using $\backslash${abbreviation}: 
              \abbreviation[As Soon As Possible]{ASAP}.
Here is an example of using $\backslash${abbreviationFull}:  
              \abbreviationFull[As Soon As Possible]{ASAP}.
\end{verbatim}

\noindent \ldots implements the following two lines.

Here is an example of using $\backslash${abbreviation}: \abbreviation[As Soon As Possible]{ASAP}.

Here is an example of using $\backslash${abbreviationFull}:  \abbreviationFull[As Soon As
Possible]{ASAP}.

\noindent If a $\backslash${listofabbreviations} command is added to the front
matter, these lines of code will automatically add two entrees to the list.

You may wish to adjust the spacing in the list of abbreviations.  To
change the spacing between the abbreviation and its definition, look
for the following lines of code in the {\em afitThesis.sty} file and
adjust ``7em'' using a smaller or larger number.

\begin{verbatim} 
 \def\l@abbreviation{\pagebreak[3]
 \vskip \lofSpace 
 \@dottedtocline{1}{0em}{7em}}
\end{verbatim}


\subsection{Creating a list of symbols}

Scientific publications may also benefit from a glossary that defines
the mathematical symbols used in the document.  Use the following
commands to implement a list of symbols:

\begin{description}
    \item[$\backslash${listofsymbols}] ~\\Produces a list of symbols
    with entries from all the $\backslash${symbol} commands in the
    body of the document.

    \item[$\backslash${symbol}[{\em definition}{]}\{{\em
    abbre}\}] ~\\Adds the {\em symbol} to text and the {\em
    symbol} and optional {\em definition} to the list of
    symbols.
\end{description}

\noindent The $\backslash${symbol} command acts like the
$\backslash${abbreviation}.

Again, you may wish to adjust the spacing in the list of symbols.  To
change the spacing between the symbol and its definition, look for the
following lines of code in the {\em afitThesis.sty} file and adjust 
``5em'' using a smaller or larger number.

\begin{verbatim}
 \def\l@symbol{\pagebreak[3]
 \vskip \lofSpace
 \@dottedtocline{1}{0em}{5em}}
\end{verbatim}