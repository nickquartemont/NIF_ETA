% !TEX root = ../SteadmanThesis.tex
\begin{abstract}
Storm enhanced densities (SEDs) are ionospheric plasma enhancements that disrupt radio communications in the near-Earth space environment, degrading the Global Positioning System (GPS) and other high-frequency systems.  Accurate GPS/total electron content (TEC) correction maps produced by ionosphere models can mitigate degradations from SEDs.  An artificial SED was created and ingested via slant TEC measurements into the Global Assimilation of Ionospheric Measurements Gauss-Markov Kalman Filter Model to determine how many ground GPS receivers are needed to produce reliable GPS/TEC correction maps over the continental United States during geomagnetic storming.  It was found that 110 well-positioned GPS receivers produced the best overall TEC accuracy, although significantly improved accuracy was still achieved if 40 or more receivers were used.  It was determined that receiver positioning had a greater impact on TEC accuracy than the number of receivers.  Additionally, it was found that TEC accuracy for the SED region increased at the expense of TEC accuracy everywhere else on the map.
\end{abstract}

