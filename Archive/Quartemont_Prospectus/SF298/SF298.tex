\date{August 2011}
\ReportDate{19--15--2011} \ReportType{Master's Thesis}
\DatesCovered{Jan 2010 --- Aug 2011}

\Title{\centering Effect of Storm Enhanced Densities on Geo-Location Accuracy Over CONUS}

%\Title{\centering \MakeUppercase{Evaluation of Interplanetary
%Magnetic Field Tracing Models Using Impulsive SEP's}}

%\ContractNumber{DACA99--99--C--9999}

%\GrantNumber{}
%\ProgramElementNumber{}
%\ProjectNumber{09ENP???}
%\TaskNumber{}
%\WorkUnitNumber{}

\Author{Capt Lindon H. Steadman}

\PerformingOrg{Air Force Institute of Technology\\[-1pt]
    Graduate School of Engineering and Management (AFIT/EN)\\[-1pt]
    2950 Hobson Way\\[-1pt]
    WPAFB OH 45433-7765}

\POReportNumber{AFIT/GAP/ENP/11-S01}

\SponsoringAgency{Air Force Weather Agency\\[-1pt]
101 Nelson Drive\\[-1pt]
Offutt AFB, NE 68113\\[-1pt]
DSN 271-0690, COMM 402-294-0690\\[-1pt]
Email: 2syosdor@offutt.af.mil }

\Acronyms{AFWA}
%\SMReportNumber{}
\DistributionStatement{APPROVED FOR PUBLIC RELEASE; DISTRIBUTION UNLIMITED.}

\Abstract{Storm enhanced densities (SEDs) are ionospheric plasma enhancements that disrupt radio communications in the near-Earth space environment, degrading the Global Positioning System (GPS) and other high-frequency systems.  Accurate GPS/total electron content (TEC) correction maps produced by ionosphere models can mitigate degradations from SEDs.  An artificial SED was created and ingested via slant TEC measurements into the Global Assimilation of Ionospheric Measurements Gauss-Markov Kalman Filter Model to determine how many ground GPS receivers are needed to produce reliable GPS/TEC correction maps over the continental United States during geomagnetic storming.  It was found that 110 well-positioned GPS receivers produced the best overall TEC accuracy, although significantly improved accuracy was still achieved if 40 or more receivers were used.  It was determined that receiver positioning had a greater impact on TEC accuracy than the number of receivers.  Additionally, it was found that TEC accuracy for the SED region increased at the expense of TEC accuracy everywhere else on the map.}

\SubjectTerms{Ionosphere, Ionosphere Forecast Model, Global Assimilation of Ionospheric Measurements Gauss-Markov Kalman Filter Model, Space Weather}

\NumberPages{110}
%\ReportClassification{}
%\PageClassification{}
%\AbstractClassification{}
\AbstractLimitation{U}

\ResponsiblePerson{Lt Col Ariel O. Acebal, AFIT/ENP}

\RPTelephone{(937) 255-3636, x4518; ariel.acebal@afit.edu}

\MakeRptDocPage
