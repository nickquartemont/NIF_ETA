% \iffalse
% Fill-in and print the Standard Form 298.
% $Id: sf298.dtx,v 1.0 99/12/14 14:15:48 sdc Exp $
% %%%%%%%%%%%%%%%%%%%%%%%%%%%%%%%%%%%%%%%%%%%%%%%%%%%%%%%%%%%%%%%%%%%%%
% This is file `sf298.ins', the installation file for the  
% `sf298' package.   
%
% Copyright (C) 2000 Steven Douglas Cochran.
%
% The sf298 package is free software; it may be distributed under
% the conditions of the LaTeX Project Public License, either
% version 1.1 of this license or (at your option) any later
% version. The latest version of this license is in
%    http://www.latex-project.org/lppl.txt
% and version 1.1 or later is part of all distributions of LaTeX 
% version 1999/06/01 or later.
%
% The sf298 package is distributed in the hope that it will be
% useful, but WITHOUT ANY WARRANTY; without even the implied
% warranty of MERCHANTABILITY or FITNESS FOR A PARTICULAR
% PURPOSE.
% %%%%%%%%%%%%%%%%%%%%%%%%%%%%%%%%%%%%%%%%%%%%%%%%%%%%%%%%%%%%%%%%%%%%%%
% @LaTeX-style-file{
%    Author     = "Steven Douglas Cochran",
%    Version    = "1.1",
%    Date       = "2000/02/29",
%    Time       = "16:26:52",
%    Filename   = "sf298.sty",
%    Address    = "Digital Mapping Laboratory, School of Computer Science
%                  Carnegie-Mellon University, 5000 Forbes Avenue
%                  Pittsburgh, PA 15213-3891, USA",
%    Telephone  = "(412) 268-5654",
%    FAX        = "(412) 268-5576",
%    Email      = "sdc+@CS.CMU.EDU (Internet)",
%    CodeTable  = "ISO/ASCII",
%    Keywords   = "LaTeX, standard forms, sf298, report documentation",
%    Supported  = "yes",
%    Abstract   = "LaTeX package for generating a completed standard
%                  form 298 (Rev.\ 8--98) as prescribed by ANSI 
%                  Std.\ Z39.18 for report documentation, for instance,
%                  as part of a document delivered on a U.S. government
%                  contract.
% }
% %%%%%%%%%%%%%%%%%%%%%%%%%%%%%%%%%%%%%%%%%%%%%%%%%%%%%%%%%%%%%%%%%%%%%%
%
% \fi
%
% \iffalse
%<*driver>
\ProvidesFile{sf298.dtx}
\documentclass{ltxdoc}
\usepackage[noconfig]{sf298}
\usepackage{afterpage}
\setlength\hfuzz{26pt}
  \IfFileExists{ltxdoc.cfg}{}{%
    \AtBeginDocument{\EnableCrossrefs\RecordChanges\CodelineIndex}
    \AtEndDocument{\PrintChanges\PrintIndex}}
\begin{document}
  \DocInput{sf298.dtx}
\end{document}
%</driver>
% \fi
%
% \changes{v1.0}{11 Jan 2000}{Initial revision.}
% %%%%%%%%%%%%%%%%%%%%%%%%%%%%%%%%%%%%%%%%%%%%%%%%%%%%%%%%%%%%%%%%%%%%%%
% \DoNotIndex{\(,\),\@car,\@date,\@empty,\@gobble,\@ifnextchar}
% \DoNotIndex{\@ifundefined,\@nil,\@tempa,\@tempboxa,\@tempskipa}
% \DoNotIndex{\@viiipt,\@vpt,\@vipt,\@xipt,\@xivpt,\@warning,\@xpt,\\}
% \DoNotIndex{\2,\AtEndDocument,\addtocounter,\advance,\arabic}
% \DoNotIndex{\baselineskip,\baselinestretch,\begin,\bf,\c@page}
% \DoNotIndex{\cleardoublepage,\csname,\date,\day,\DeclareFixedFont}
% \DoNotIndex{\def,\documentclass,\documentpackage,\edef,\else,\end}
% \DoNotIndex{\endcsname,\evensidemargin,\ExecuteOptions,\expandafter}
% \DoNotIndex{\fi,\font,\fontdimen,\footnote,\framebox,\global,\hbox}
% \DoNotIndex{\headheight,\headsep,\hskip,\hss,\if,\if@twoside,\ifdim}
% \DoNotIndex{\IfFileExists,\ifodd,\ifx,\InputIfFileExists,\LARGE}
% \DoNotIndex{\large,\let,\line,\lineskip,\linethickness,\makeatletter}
% \DoNotIndex{\makeatother,\makebox,\maketitle,\month,\NeedsTeXFormat}
% \DoNotIndex{\newcommand,\newcounter,\newif,\newpage,\noindent}
% \DoNotIndex{\normalbaselineskip,\null,\number,\oddsidemargin}
% \DoNotIndex{\pagenumbering,\pagestyle,\par,\parbox,\phantom}
% \DoNotIndex{\ProvidesPackage,\put,\r@TotPages,\raggedright,\relax}
% \DoNotIndex{\renewcommand,\rule,\section,\setcounter,\setbox}
% \DoNotIndex{\setlength,\shortstack,\space,\spaceskip,\string}
% \DoNotIndex{\tableofcontents,\textheight,\textwidth,\thispagestyle}
% \DoNotIndex{\topmargin,\tt,\typeout,\unitlength,\usepackage,\value}
% \DoNotIndex{\vbox,\vfil,\voffset,\vskip,\vss,\wd,\xspaceskip,\year}
%
% \newcommand{\regsf}[1]{{\upshape\mdseries\sffamily #1}}
% \newcommand{\eg}{{\it e.g.\/}}
% \newcommand{\ie}{{\it i.e.\/}}
% \def\filename{sf298.dtx}
% \def\fileversion{1.1}
% \def\filedate{2000/02/29}
% \def\docdate {2000/03/03}
% \makeatletter
%   \def\drawform{\sf@drawform}
%   \def\@oddfoot{%
%     \rm
%     \hfil
%     \@ifundefined{r@fig:form}%
%       {thepage}%
%       {\@tempcnta\pageref{fig:form}
%        \ifnum \c@page=\@tempcnta \else
%          \thepage
%        \fi}%
%     \hfil}
%   \let\@evenfoot\@oddfoot
% \makeatother
%
% \CheckSum{1135}
%
% \CharacterTable
%  {Upper-case    \A\B\C\D\E\F\G\H\I\J\K\L\M\N\O\P\Q\R\S\T\U\V\W\X\Y\Z
%   Lower-case    \a\b\c\d\e\f\g\h\i\j\k\l\m\n\o\p\q\r\s\t\u\v\w\x\y\z
%   Digits        \0\1\2\3\4\5\6\7\8\9
%   Exclamation   \!     Double quote  \"     Hash (number) \#
%   Dollar        \$     Percent       \%     Ampersand     \&
%   Acute accent  \'     Left paren    \(     Right paren   \)
%   Asterisk      \*     Plus          \+     Comma         \,
%   Minus         \-     Point         \.     Solidus       \/
%   Colon         \:     Semicolon     \;     Less than     \<
%   Equals        \=     Greater than  \>     Question mark \?
%   Commercial at \@     Left bracket  \[     Backslash     \\
%   Right bracket \]     Circumflex    \^     Underscore    \_
%   Grave accent  \`     Left brace    \{     Vertical bar  \|
%   Right brace   \}     Tilde         \~}
%
% \title{The \regsf{sf298} Package\footnote{This paper documents
%        the \regsf{sf298} package v\fileversion, last revised \filedate.}}
% \author{Steven Douglas Cochran\\[5pt]
%         Digital Mapping Laboratory, School of Computer Science \\
%         Carnegie-Mellon University, 5000 Forbes Avenue \\
%         Pittsburgh, PA 15213--3891, USA\\[5pt]
%         \texttt{sdc+@cs.cmu.edu}}
% \date{\docdate}
% \maketitle
% \vspace{-20pt}
% \begin{abstract}
% \noindent
% This article documents the \LaTeX\ package for generating a completed
% standard form 298 (Rev.\ 8--98) as prescribed by ANSI Std.\ Z39.18 for
% report documentation, for instance, as part of a document delivered on
% a U.S. government contract.
%
%\end{abstract}
% 
% \setcounter{tocdepth}{2}
% \tableofcontents
%
% \section{Introduction}
%   \begin{figure}
%     \ReportDate{}
%     \DistributionStatement{}
%     \ReportClassification{}
%     \AbstractClassification{}
%     \PageClassification{}
%     \AbstractLimitation{}
%     \NumberPages{}
%     \Abstract{\hfill Figure~\ref{fig:form}: Blank Special Form 298
%               (Report Documentation Page).\hfill\null}
%     \refstepcounter{figure}
%     \label{fig:form}
%     \vbox to \textheight{%
%       \vspace{-78pt}%
%       \hbox to \textwidth{%
%         \hspace{-130pt}%
%         \drawform
%         \hss}%
%       \vss}
%   \end{figure}
% 
% The \regsf{sf298} package provides for creating a filled-in copy of
% the standard form 298 (Rev.\ 8--98), ``Report Documentation Page''
% (shown in Figure~\ref{fig:form}).  This form is used in announcing and
% cataloging reports submitted as deliverables on contracts with the
% U.S. government.  It is important that the information on the sf298
% page be consistent with the rest of the report, particularly the cover
% and title page.  Instructions for filling in each block of the form
% are given below.  The macro \cmd{\MakeRptDocPage} causes the page to be
% printed.  If the \regsf{twoside} option is specified in the
% documentclass, then it is printed as a separate page with a blank
% back.
% 
% If you are changing the base font families for your paper (\eg, you
% load the \regsf{times} package to use the PostScript fonts) and you
% want the fonts used on the |sf298| to change also, then you must
% change the font families before loading the \regsf{sf298} package (see
% section~\ref{sec:configfont}).
% 
% The package takes two options, the default |config|, causes it to
% check for a configuration file and the other |noconfig|, disables
% this check (see section~\ref{sec:configfile}).
%
% A sample report is given in section~\ref{sec:sample} that shows one
% approach to using the \regsf{sf298} package.
% 
% \section{The individual fields}
% 
% There are nineteen fields, three with multiple sub-fields, on the
% sf298.  The following macros, each having a single argument, define
% the data to be included in each field.  Usually only a few fields will
% apply to a given report.  Table~\ref{tab:attributes} notes which
% fields are required and which have default values.  It also gives an
% approximate number of lines and characters per line that will fit into
% each block (for the Computer Modern Roman font family, PostScript Times
% Roman or other fonts will differ).  The best approach to insure that 
% the form is properly filled-in is to preview the page and adjust any
% fields that appear to have a problem.
% 
% A few fields allow a partial extra line along the top of the box.  The
% last two columns of Table~\ref{tab:attributes} shows the offset needed
% to add such an extra line.  To use this extra space, you must already
% have used up the maximum number of (regular) lines for this field and
% you must begin the argument with an \cmd{\hspace} of the size of the
% indicated {\bf Skip}.  The last {\bf Chars.} column gives the approximate
% amount of extra characters that you can add.
% 
% \begin{table}
% \null\vspace{-.1in}
% \def\n{\hfill{\bf ---}\hfill\null}
% \def\d#1#2{\smash{\lower#1\hbox{#2}}}
% \def\p{\phantom{0}}
% \def\s#1{{\footnotesize (#1)}}
% \def\z#1{\hbox to 0pt{#1\hss}\p\p\p}
% \doublerulesep.5pt
% \arrayrulewidth.5pt
% \caption{Field Attributes and Approximate Maximum Size.}
% \label{tab:attributes}
% \vspace{10pt}
% \DeleteShortVerb{\|}
% \hbox to 1.1\textwidth{%
% \hss
% \begin{tabular}{rlcccccc}                                                           \hline\hline
%   \multicolumn{1}{c}{\rule{0pt}{14pt}\bf Field}
%           & \multicolumn{1}{c}{\d{1ex}{\bf Macro Name}}
%                                           & \multicolumn{1}{c}{\d{1ex}{\bf Required}}
%                                                 & \multicolumn{1}{l}{\bf Default}
%                                                       & \multicolumn{1}{c}{\bf Max}
%                                                            & \multicolumn{1}{c}{{\bf Chars.}\d{-2.5pt}{\s{7}}}
%                                                                    & \multicolumn{2}{c}{\d{-2pt}{{\bf Extra Space}\d{-2.5pt}{\s{9}}}}
%                                                                                   \\ \cline{7-8}
%   \multicolumn{1}{c}{\rule[-7pt]{0pt}{7pt}\bf Number}
%           &                               &     & \multicolumn{1}{l}{{\bf Value}\d{-2.5pt}{\s{1}}}
%                                                       & \multicolumn{1}{c}{\bf Lines}
%                                                             & \multicolumn{1}{c}{\bf per Line}
%                                                                     & \multicolumn{1}{c}{\d{2pt}{{\bf Skip}\d{-2.5pt}{\s{8}}}}
%                                                                            & \multicolumn{1}{c}{\d{2pt}{{\bf Chars.}\d{-2.5pt}{\s{7}}}}
%                                                                                   \\ \hline\rule{0pt}{14pt}
%    1\z{}  & \cmd{\ReportDate}             & Yes &\s{2}& \p1 &  \p31 & \n   & \n   \\
%    2\z{}  & \cmd{\ReportType}             & Yes & \n  & \p1 &  \p46 & \n   & \n   \\
%    3\z{}  & \cmd{\DatesCovered}           & Yes & \n  & \p1 &  \p39 & \n   & \n   \\
%    4\z{}  & \cmd{\Title}                  & Yes & \n  & \p7 &  \p72 & 35mm & \p48 \\
%    5\z{a} & \cmd{\ContractNumber}         & \n  & \n  & \p1 &  \p46 & \n   & \n   \\
%    5\z{b} & \cmd{\GrantNumber}            & \n  & \n  & \p1 &  \p46 & \n   & \n   \\
%    5\z{c} & \cmd{\ProgramElementNumber}   & \n  & \n  & \p1 &  \p46 & \n   & \n   \\
%    5\z{d} & \cmd{\ProjectNumber}          & \n  & \n  & \p1 &  \p46 & \n   & \n   \\
%    5\z{e} & \cmd{\TaskNumber}             & \n  & \n  & \p1 &  \p46 & \n   & \n   \\
%    5\z{f} & \cmd{\WorkUnitNumber}         & \n  & \n  & \p1 &  \p46 & \n   & \n   \\
%    6\z{}  & \cmd{\Author}                 & Yes & \n  & \p7 &  \p72 & 22mm & \p58 \\
%    7\z{}  & \cmd{\PerformingOrg}          & Yes & \n  & \p4 &  \p81 & \n   & \n   \\
%    8\z{}  & \cmd{\POReportNumber}         & \n  & \n  & \p4 &  \p39 & 18mm & \p26 \\
%    9\z{}  & \cmd{\SponsoringAgency}       & Yes & \n  & \p5 &  \p81 & \n   & \n   \\
%   10\z{}  & \cmd{\Acronym}                & \n  & \n  & \p2 &  \p30 & \n   & \n   \\
%   11\z{}  & \cmd{\SMReportNumber}         & \n  & \n  & \p1 &  \p39 & 23mm & \p23 \\
%   12\z{}  & \cmd{\DistributionStatement}  & \n  &\s{3}& \p3 &   120 & 70mm & \p75 \\
%   13\z{}  & \cmd{\SupplementaryNotes}     & \n  & \n  & \p2 &   120 & 45mm & \p92 \\
%   14\z{}  & \cmd{\Abstract}               & Yes & \n  &  10 &   120 & 22mm &  105 \\
%   15\z{}  & \cmd{\SubjectTerms}           & \n  & \n  & \p3 &   125 & 32mm &  103 \\
%   16\z{a} & \cmd{\ReportClassification}   & Yes &\s{4}& \p1 & \p\p9 & \n   & \n   \\
%   16\z{b} & \cmd{\AbstractClassification} & Yes &\s{4}& \p1 & \p\p9 & \n   & \n   \\
%   16\z{c} & \cmd{\PageClassification}     & Yes &\s{4}& \p1 & \p\p9 & \n   & \n   \\
%   17\z{}  & \cmd{\AbstractLimitation}     & \n  &\s{5}& \p1 &  \p12 & \n   & \n   \\
%   18\z{}  & \cmd{\NumberPages}            & Yes &\s{6}& \p1 & \p\p8 & \n   & \n   \\
%   19\z{a} & \cmd{\ResponsiblePerson}      & \n  & \n  & \p1 &  \p41 & \n   & \n   \\ \rule[-7pt]{0pt}{7pt}
%   19\z{b} & \cmd{\RPTelephone}            & \n  & \n  & \p1 &  \p41 & \n   & \n   \\ \hline\hline
%   \multicolumn{8}{l}{\rule{0pt}{14pt}{\bf Notes:}}                                \\
%   \multicolumn{8}{l}{\quad\s{1} All defaults may be overridden by specifying a 
%                      value.  Custom defaults may be assigned using a}             \\
%   \multicolumn{8}{l}{\quad\phantom{\s{1}} configuration file (see 
%                      section~\ref{sec:configfile})}                               \\
%   \multicolumn{8}{l}{\quad\s{2} Defaults to the current date (in dd-mm-yyyy 
%                      format).}                                                    \\
%   \multicolumn{8}{l}{\quad\s{3} Defaults to ``Approved for public release; 
%                      distribution is unlimited''}                                 \\
%   \multicolumn{8}{l}{\quad\s{4} Defaults to ``U'' (Unclassified)}                 \\
%   \multicolumn{8}{l}{\quad\s{5} Defaults to ``UU'' (Unclassified Unlimited).}     \\
%   \multicolumn{8}{l}{\quad\s{6} If the \regsf{totpages} package is present, then 
%                      the total number of pages is calculated automatically,}      \\
%   \multicolumn{8}{l}{\quad\phantom{\s{1}} otherwise, there is no default (see 
%                      section~\ref{sec:configpage}).}                              \\
%   \multicolumn{8}{l}{\quad\s{7} The number of characters is approximate and for
%                      the Computer Modern Roman font family only.}                 \\
%   \multicolumn{8}{l}{\quad\s{8} Amount to ident the extra first line, 
%                      \eg, \cmd{\Title}\{\cmd{\hspace}\{37mm\} \dots text\dots\}.} \\
%   \multicolumn{8}{l}{\quad\s{9} More text may be placed in these blocks by
%                      adjusting the font family or size and/or changing}           \\
%   \multicolumn{8}{l}{\quad\phantom{\s{9}} the interline gap (see \cmd{\PerformingOrg} in
%                      sections~\ref{sec:configfile} and \ref{sec:sample} for an example).}
% \end{tabular}}
% \vspace{.1in}
% \MakeShortVerb{\|}
% \end{table}
% 
% For those blocks that allow more than one line (see
% Table~\ref{tab:attributes}), the text argument will automatically be
% broken to fit the box and the paragraph is usually vertically centered
% and left justified.  Use can use ``|\\|'' (newline) to override the
% default linebreaking, if desired (see section~\ref{sec:configfile} for
% an example).
%
% \paragraph{1. REPORT DATE.}
% \DescribeMacro{\ReportDate}
%
% \noindent
% Full publication date including day, and month if available.
% Must cite at least the year and be Year 2000 compliant, \eg
% \par\null\noindent
% \begin{tabular}{llll}
%   ~~ & {\tt \cmd{\ReportDate}\{30--06--1998\}} & $\rightarrow$ & 30--06--1998 \\
%   ~~ & {\tt \cmd{\ReportDate}\{xx--06--1998\}} & $\rightarrow$ & xx--06--1998 \\
%   ~~ & {\tt \cmd{\ReportDate}\{xx--xx--1998\}} & $\rightarrow$ & xx--xx--1998
% \end{tabular}
% \par\null\noindent
% This field is required, but will default to the current date.
%
% \paragraph{2. REPORT TYPE.}
% \DescribeMacro{\ReportType}
%
% \noindent
% State the type of report, such as final, technical, interim,
% memorandum, master's thesis, progress, quarterly, research, special,
% group study, {\it etc\@.\/}  This field is required.
% 
% \paragraph{3. DATES COVERED.}
% \DescribeMacro{\DatesCovered}
%
% \noindent
% Indicate the time during which the work was performed and the report
% was written, \eg,
% \par\null\noindent
% \begin{tabular}{llll}
%   ~~ & {\tt \cmd{\DatesCovered}\{Jun 1997--Jun 1998\}} & $\rightarrow$ & Jun 1997--Jun 1998 \\
%   ~~ & {\tt \cmd{\DatesCovered}\{1--10 Jun 1996\}}     & $\rightarrow$ & 1--10 Jun 1996     \\
%   ~~ & {\tt \cmd{\DatesCovered}\{May--Nov 1998\}}      & $\rightarrow$ & May--Nov 1998      \\
%   ~~ & {\tt \cmd{\DatesCovered}\{Nov 1998\}}           & $\rightarrow$ & Nov 1998
% \end{tabular}
% \par\null\noindent
% The month and year for at least one month are required.
% 
% \paragraph{4. TITLE AND SUBTITLE.}
% \DescribeMacro{\Title}
%
% \noindent
% A title is taken from the part of the report that provides the most
% meaningful and complete information.  When a report is prepared in
% more than one volume, repeat the primary title, add volume number, and
% include subtitle for the specific volume.  On classified documents enter
% the title classification in parentheses.  This field is required.
% 
% \paragraph{5a. CONTRACT NUMBER.}
% \DescribeMacro{\ContractNumber}
%
% \noindent
% Enter all contract numbers as they appear in the report, \eg
%
% \par\null\noindent
% \begin{tabular}{llll}
%   ~~ & {\tt \cmd{\ContractNumber}\{F33615--86--C--5169\}} & $\rightarrow$ & F33615--86--C--5169
% \end{tabular}
%
% \paragraph{5b. GRANT NUMBER.}
% \DescribeMacro{\GrantNumber}
%
% \noindent
% Enter all grant numbers as they appear in the report, \eg
%
% \par\null\noindent
% \begin{tabular}{llll}
%   ~~ & {\tt \cmd{\GrantNumber}\{AFOSR--82--1234\}} & $\rightarrow$ & AFOSR--82--1234
% \end{tabular}
%
% \paragraph{5c. PROGRAM ELEMENT NUMBER.}
% \DescribeMacro{\ProgramElementNumber}
%
% \noindent
% Enter all program element numbers as they appear in the report, \eg
%
% \par\null\noindent
% \begin{tabular}{llll}
%   ~~ & {\tt \cmd{\ProgramElementNumber}\{61101A\}} & $\rightarrow$ & 61101A
% \end{tabular}
%
% \paragraph{5d. PROJECT NUMBER.}
% \DescribeMacro{\ProjectNumber}
%
% \noindent
% Enter all project numbers as they appear in the report, \eg
%
% \par\null\noindent
% \begin{tabular}{llll}
%   ~~ & {\tt \cmd{\ProjectNumber}\{1F665702D1257; ILIR\}} & $\rightarrow$ & 1F665702D1257; ILIR
% \end{tabular}
%
% \paragraph{5e. TASK NUMBER.}
% \DescribeMacro{\TaskNumber}
%
% \noindent
% Enter all task numbers as they appear in the report, \eg
%
% \par\null\noindent
% \begin{tabular}{llll}
%   ~~ & {\tt \cmd{\TaskNumber}\{05; RF0330201; T4112\}} & $\rightarrow$ & 05; RF0330201; T4112
% \end{tabular}
%
% \paragraph{5f. WORK UNIT NUMBER.}
% \DescribeMacro{\WorkUnitNumber}
%
% \noindent
% Enter all work unit numbers as they appear in the report, \eg
%
% \par\null\noindent
% \begin{tabular}{llll}
%   ~~ & {\tt \cmd{\WorkUnitNumber}\{001; AFAPL30480105\}} & $\rightarrow$ & 001; AFAPL30480105
% \end{tabular}
%
% \paragraph{6. AUTHOR(S).}
% \DescribeMacro{\Author}
%
% \noindent
% Enter name(s) of the person(s) responsible for writing the report,
% performing the research, or credited with the content of the report.
% The form of entry is the last name, first name, middle initial, and
% additional qualifiers separated by commas, \eg
% \par\null\noindent
% \begin{tabular}{llll}
%   ~~ & {\tt \cmd{\Author}\{Smith, Richard, J, Jr\}} & $\rightarrow$ & Smith, Richard, J, Jr
% \end{tabular}
% \par\null\noindent
% If editor or compiler, this should follow the name(s).
%
% \paragraph{7. PERFORMING ORGANIZATION NAME(S) AND ADDRESS(ES).}
% \DescribeMacro{\PerformingOrg}
%
% \noindent
% Self-explanatory.  This field is one of the best examples of one that
% may be defined in a configuration file that may be used for many
% papers.  For an example of how to use a configuration file, see
% section~\ref{sec:configfile}.
% 
% \paragraph{8. PERFORMING ORGANIZATION REPORT NUMBER.}
% \DescribeMacro{\POReportNumber}
%
% \noindent
% Enter the unique alphanumeric report number(s) assigned by the
% performing organization, \eg
% \par\null\noindent
% \begin{tabular}{llll}
%   ~~ & \multicolumn{3}{l}{\tt \cmd{\POReportNumber}\{BRL--1234; AFWL--TR--85--4017--Vol21--PT--2\}} \\ 
%      & \hspace{1in} & $\rightarrow$ & BRL--1234; AFWL--TR--85--4017--Vol21--PT--2
% \end{tabular}
% 
% \paragraph{9. SPONSORING/MONITORING AGENCY NAME(S) AND ADDRESS(ES).}
% \DescribeMacro{\SponsoringAgency}
%
% \noindent
% Enter the name and address of the organization(s) financially
% responsible for and monitoring the work.  If it is desired to place
% two organizations in this field, then they may be split up in two
% different ways, the first is to use two 2-line addresses with (for
% example) a % ``|\\[10pt]|'' line break after the first address to show
% both % the sponsoring and the monitoring agency addresses.  The second
% method is to place the addresses in a |tabular| environment.  An 
% example of both methods is shown below.
% \begin{verbatim}
%   \SponsoringAgency{%
%     U.S. Government Sponsoring Department    \\
%     P.O. Box 92919, Miltown, AK  97909--6543 \\[10pt]
%     U.S. Army Monitoring Agency              \\
%     909 Mission Rd., Somecity, MD  29399--9753}
%   
%   \SponsoringAgency{%
%     \begin{tabular}{@{}l@{\hspace{5mm}}l@{}}
%       U.S. Government Sponsoring Dept. & U.S. Government Monitoring Agency \\
%       P.O. Box 92919                   & 909 Mission Rd.                   \\
%       Miltown, AK  97909--6543         & Somecity, MD  29399--9753
%     \end{tabular}}
% \end{verbatim}
% These give the following, respectively:
% \par\null\noindent 
% \frame{\vbox to 72.8pt{%
%   \vfil
%   \parbox{364.2pt}{%
%     U.S. Government Sponsoring Dept.         \\
%     P.O. Box 92919, Miltown, AK  97909--6543 \\[10pt]
%     U.S. Government Monitoring Agency        \\
%     909 Mission Rd., Somecity, MD  29399--9753}%
%   \vfil}}
% \par\null\noindent 
% \frame{\vbox to 72.8pt{%
%   \vfil
%   \parbox{364.2pt}{%
%     \begin{tabular}{@{}l@{\hspace{5mm}}l@{}}
%       U.S. Government Sponsoring Dept. & U.S. Government Monitoring Agency \\
%       P.O. Box 92919                   & 909 Mission Rd.                   \\
%       Miltown, AK  97909--6543         & Somecity, MD  29399--9753
%     \end{tabular}}
%   \vfil}}
%
% \paragraph{10. SPONSOR/MONITOR'S ACRONYM(S).}
% \DescribeMacro{\Acronyms}
%
% \noindent
% Enter any acronym(s) that pertain to the Sponsor/Monitor, separated
% with commas, if available, \eg
% \par\null\noindent
% \begin{tabular}{llll}
%   ~~ & {\tt \cmd{\Acronyms}\{BRL, ARDEC, NADC\}} & $\rightarrow$ & BRL, ARDEC, NADC
% \end{tabular}
%
% \paragraph{11. SPONSOR/MONITOR'S REPORT NUMBER(S).}
% \DescribeMacro{\SMReportNumber}
%
% \noindent
% Enter the report number(s) as assigned by the sponsoring/monitoring
% agency, if available, \eg
% \par\null\noindent
% \begin{tabular}{llll}
%   ~~ & {\tt \cmd{\SMReportNumber}\{BRL-TR--829\}} & $\rightarrow$ & BRL-TR--829
% \end{tabular}
%
% \newpage
% \paragraph{12. DISTRIBUTION/AVAILABILITY STATEMENT.}
% \DescribeMacro{\DistributionStatement}
%
% \noindent
% Use agency-mandated availability statements to indicated the public
% availability or distribution limitations of the report.  If additional
% limitations/restrictions or special markings are indicated follow
% agency authorization procedures, \eg, RD/FRD, PROPIN, ITAR.
% Include copyright information. This field is {\bf not} required, but will
% default to the first of the following examples:
% \par\null\noindent
% ``Approved for public release; distribution is unlimited.''
% \par\null\noindent
% ``Distribution authorized to DOD only; (reason and date).  Other
% requests shall be referred to (controlling DOD office).''
% \par\null\noindent
% ``Further dissemination only as directed by (controlling DOD office
% and date) or higher DOD authority.''
% \par\null\noindent
% \hbox to \textwidth{%
%   \begin{tabular}{l@{\space---\space}l}
%     For DOD  & See DODD 5230.24, ``Distribution Statements on Technical
%                                    Documents.''\\
%     For DoE  & See authorities.\\
%     For NASA & See Handbook NHB 2200.2.\\
%     For NTIS & Leave blank.
%   \end{tabular}%
%   \hss}
%
% \paragraph{13. SUPPLEMENTARY NOTES.}
% \DescribeMacro{\SupplementaryNotes}
%
% \noindent
% Enter information not included elsewhere such as: ``Prepared in
% cooperation with\dots''; ``Translation of\dots''; ``To be published
% in\dots''.  When a report is revised, include a statement whether the
% new report supersedes or supplements the older report and the older
% report number if different.  ARO requires the following text:
% \begin{verse}\noindent
%   ``The views, opinions and/or findings contained in this report
%   are those of the authors and should not be construed as an
%   official Department of Army position, policy or decision, unless
%   so designated by other documentation.''
% \end{verse}
% 
% \paragraph{14. ABSTRACT.}
% \DescribeMacro{\Abstract}
%
% \noindent
% Include a brief (approximately 200 words) factual summary of
% the most significant information contained in the report.  This field 
% is required.
% 
% \paragraph{15. SUBJECT TERMS.}
% \DescribeMacro{\SubjectTerms}
%
% \noindent
% Enter keywords or phrases identifying major subjects in the report.
% Separate key words and phrases with semicolons (;).  
% 
% \paragraph{16a. REPORT SECURITY CLASSIFICATION.}
% \DescribeMacro{\ReportClassification}
%
% \noindent
% Enter U.S.\ Security Classification in accordance with U.S.\ Security
% Regulations, \eg\ U, C, S, etc.  If this form contains classified 
% information, stamp classification level on the top and bottom of the 
% page.  This field is required, but will default to ``U'' (Unclassified).
% 
% \paragraph{16b. ABSTRACT SECURITY CLASSIFICATION.}
% \DescribeMacro{\AbstractClassification}
%
% \noindent
% See 16a (above).
%
% \paragraph{16c. THIS PAGE SECURITY CLASSIFICATION.}
% \DescribeMacro{\PageClassification}
%
% \noindent
% See 16a (above).
%
% \paragraph{17. LIMITATION OF ABSTRACT.}
% \DescribeMacro{\AbstractLimitation}
%
% \noindent
% This block must be completed to assign a distribution limitation to
% the abstract.  Enter either ``UU'' (Unclassified Unlimited) or ``SAR''
% (same as report).  An entry in this block is necessary if the abstract
% is to be limited.  If blank, the abstract is assumed to be unlimited.
% This field is {\bf not} required, but defaults to ``UU''.
% 
% \paragraph{18. NUMBER OF PAGES.}
% \DescribeMacro{\NumberPages}
% \DescribeMacro{\NumberPages*}
%
% \noindent
% Enter the total number of pages (including including Roman numerals).
% This field is required.  If the \regsf{totpages} package\footnote{In
% addition to the \regsf{totpages} package, you will need the
% \regsf{everyshi} package ({\sc\scriptsize EVERY SHIpout}) that is used 
% by the \regsf{totpages} package.} is available at your site, then you 
% may ignore this field and a value will be entered on the page after the
% processing of the document with \LaTeX.  As with labels and citations
% referenced in the document, a message will be printed if \LaTeX\ must
% be run again to propagate a correct label value to the sf298 form.  If
% a value is given with this macro, then it {\bf will} be the one used
% on the form.
% 
% There is also a star version of the \cmd{\NumberPages} macro that is 
% only used with the \regsf{totpages} package.  This adds the given 
% number of pages to the result rather than directly setting the page 
% number.  This macro is used when you are inserting a known number of 
% pages, say for an appendix, into the document.
% 
% \paragraph{19a. NAME OF RESPONSIBLE PERSON.}
% \DescribeMacro{\ResponsiblePerson}
%
% \noindent
% Enter the name of the person responsible for submission of the
% document.
%
% \paragraph{19b. TELEPHONE NUMBER OF RESPONSIBLE PERSON.}
% \DescribeMacro{\RPTelephone}
%
% \noindent
% Enter the telephone number of the person responsible for the
% submission of the document (including area code).
%
% \section{Configuration}
% \enlargethispage{24pt}
% 
% In order to make the use of the report documentation page as easy as
% possible, several aspects of the package are automated or can be easily
% configured.
% 
% \subsection{Fonts}
% \label{sec:configfont}
% 
% When the package is loaded it defines the fonts that are used by
% basing them on the font families \cmd{\sfdefault} for the field labels and
% \cmd{\rmdefault} for the text entries.  By default, they will be based on
% the Computer Modern font family.  If you are changing the base font
% families for your paper and you want the fonts used on the |sf298| to
% change also, then you must change the font families before loading the
% \regsf{sf298} package.  See \cite[pages 229,236]{Kopk-Daly93} for more 
% information about the default font families.
% 
% For any specific field, you can override the default font by 
% declaring your preferred value along with your entry, \eg
% \par\null\noindent 
% \begin{tabular}{llll}
%   ~~ & {\tt \cmd{\Title}\{\cmd{\Huge} A Huge Title\}} & $\rightarrow$ & \smash{\Huge\lower.5ex\hbox{A Huge Title}}
% \end{tabular}
%
% \subsection{Configuration file}
% \label{sec:configfile}
% 
% When the package is loaded, it searches the along the standard \TeX\
% search path, as configured in the system and modified by 
% the {\tt TEXINPUTS} environment variable (or equivalent) for the
% for the file ``{\tt sf298.cfg}''.  It loads the first such file that
% it finds and this file may be used to change defaults or make other
% changes to the \regsf{sf298} package, \eg
% \par\null\noindent 
% \begin{tabular}{ll@{}l}
%   ~~ & {\tt \cmd{\PerformingOrg}\{} & {\tt Our Little Laboratory \cmd{\\[-1pt]}}\footnotemark \\[-1pt]
%      &                              & {\tt Computer Science Department \cmd{\\[-1pt]}}        \\[-1pt]
%      &                              & {\tt The Big University \cmd{\\[-1pt]}}                 \\[-1pt]
%      &                              & {\tt 9999 Avenue Street \cmd{\\[-1pt]}}                 \\[-1pt]
%      &                              & {\tt Anytown, TX  19293--8765}
% \end{tabular}\footnotetext{Note that five lines are fit where only four are recommended in 
%                            Table~\ref{tab:attributes} by shrinking the baselineskip by one point.}
% \par\null\noindent 
% For an organization, other typical fields for inclusion in the
% configuration file are \cmd{\ResponsiblePerson} and
% \cmd{\RPTelephone}.  For a set of papers that are part of the same
% contract, you could set the \TeX\ search path so that a local version
% of {\tt sf298.cfg} is loaded that first \cmd{\inputs} the more global
% one and then sets contract specific defaults associated with a single
% contract, \eg, \cmd{\ContractNumber}, \cmd{\Sponsoring Agency}, 
% \cmd{\Acronym}, {\it etc.\/}
%
% The option |noconfig| turns off the input of the configuration file.
% For symmetry, there exists an option |config|, but that is the 
% default and does not need to be specified.
%
% \subsection{``Number of Pages'' field}
% \label{sec:configpage}
% 
% If the \regsf{totpages} package is available, then the {\bf Number of
% Pages} field will be automatically filled-in by calculating the
% difference between the page count of the |sf298| page and the last
% page.  If this package is not present, and if the field
% is not defined, a warning will be issued.
% 
% If the {\bf Number of Pages} is incorrectly calculated, then the
% correct value can be explicitly defined with the \cmd{\NumberPage}
% macro or offset by a fixed value with the \cmd{\NumberPage*} macro.
% The former may be needed if the sf298 form is not the first page of
% the document and an example of the latter is where there is an
% appendix whose pages are inserted and are, therefore, not
% automatically counted.
% 
% \subsection{Adjusting the form placement}
% \DescribeMacro{\LeftShift}
% \DescribeMacro{\DownShift}
% \changes{v1.1}{15 Feb 2000}{Changed reference of millimeters to big points.}
% 
% \noindent
% Sometimes you need to adjust the placement of a large form, such as to
% provide an extra amount of margin space or for a different page size.
% These two macros allow this fine-tuning: \cmd{\LeftShift\{5\}} will cause
% the form to be shifted 5 big points to the left\@.\footnote{Note that 
% there are no units here, the unit step is defined as one big point 
% (1 bp) at the beginning of the \cmd{\MakeRptDocPage} command above by 
% setting the \cmd{\unitlength}.}  The \cmd{\DownShift\{-8.5\}} will cause the 
% form to be shifted up $8\frac{1}{2}$ big points (\ie, down a negative 
% $8\frac{1}{2}$ big points).  For example, the following commands will 
% center the sf298 form on A4 paper:
% \par\null\noindent
% \begin{tabular}{ll}
%   ~~ & {\tt \cmd{\LeftShift}\{8.5\}} \\
%      & {\tt \cmd{\DownShift}\{-22.0\}}
% \end{tabular}
%
% \section{An example of use with the \regsf{article} class}
% \label{sec:sample}
% \iffalse
%<*sample>
% \fi%
% 
% This sample program demonstrates how the \regsf{sf298} package may 
% be used and integrated with an existing class.  Here we redefine 
% \cmd{\@makefile} so that we can create a titlepage and report
% documentation page that share a common set of fields, such as the
% report title, the performing organization title and address and the
% report abstract.
% 
% First we have start the paper as an \regsf{article} with twosided
% pages and using the \regsf{sf298} package.  We also make the `@' a
% letter so that we can update some internal names.
% 
%    \begin{macrocode}
\documentclass[twoside]{article}
\usepackage{sf298}
\makeatletter
%    \end{macrocode}
%
% \begin{macro}{\@maketitle}
% 
% We modify the standard \cmd{\@maketitle} so that the title and author's
% address (from the internal \cmd{\sf@PeformingOrg} value) are
% integrated into the standard titlepage layout.  This may not always be
% the best approach, since the blocks may contain different internal
% layout information on the title page and on the report documentation
% page.  Here, the \cmd{\@date} macro is used for the date rather than
% \cmd{\sf@ReportDate} and similarly for the author's names, since the
% desired format is different for these fields, even though the
% information is the same.  Another way to do the same thing is to
% define the macro to parse and format these values as desired in each
% case.
% 
% As much as possible, a document class that generates both a titlepage
% and the report documentation page (sf298) should allow the common
% information to be entered only once.
%
%    \begin{macrocode}
\renewcommand{\@maketitle}{%
  \newpage
  \null
  \vskip 2em%
  \begin{center}%
  \let\footnote\thanks
    {\LARGE\sf@Title\par}%
    \vskip 1.5em%
    {\large
      \lineskip .5em%
      \begin{tabular}[t]{c}%
        \@author\\[10pt]
        \sf@PerformingOrg
      \end{tabular}\par}%
    \vskip 1em%
    {\large\@date}%
  \end{center}%
  \par
  \vskip 1.5em}
%    \end{macrocode}
% \end{macro}
%
% \begin{macro}{\showabstract}
% 
% The next macro, \cmd{\showabstract} shows one method of pulling the
% abstract text out of the internal (``hidden'') define, in order to
% include it on the titlepage as we show in the accompanying
% ``sample298.txt'' file.  We just have to make sure that we use it
% before making the report documentation page, since this data is
% thrown away after that.
%
%    \begin{macrocode}
 \newcommand{\showabstract}{%
   \noindent
   \sf@Abstract}
%    \end{macrocode}
% \end{macro}
%
% \begin{macro}{\ReportDescription}
% 
% At the end of the \cmd{\titlepage} command, the \cmd{\thanks} macro is
% redefined to be \cmd{\relax}.  If we want to define the title once and
% also want to be able to have a \cmd{\thanks} message associated with the
% title on the titlepage, and want to use that same title text on the
% report documentation page, then we need to redefine the \cmd{\thanks}
% macro to eat it's argument.  This is done by setting \cmd{\thanks} to
% \cmd{\@gobble} while printing the report documentation page.
% 
%    \begin{macrocode}
\newcommand{\ReportDescription}{%
   \global\let\thanks\@gobble
   \MakeRptDocPage
   \global\let\thanks\relax}
%    \end{macrocode}
% \end{macro}
%
% We are finished with our example integration, so we close off changing
% the internals and make `@' an ``other'' once more.
%
%    \begin{macrocode}
\makeatother
%    \end{macrocode}
% 
% Now we fill in the fields needed for the title and the report
% documentation pages.  For reference as a template, we have entered all
% of the \regsf{sf298} block macros below.  The ones that are not being
% used are commented out.  Note that although block 7 (Performing Organization Name) 
% is rated at four lines according to Table~\ref{tab:attributes}, we can fit five
% lines if we compress them a little.  Depending on your base font family and
% such small adjustments, you can probably fit any reasonable set of information
% onto this form in the way that you want to.
%
%    \begin{macrocode}
\date{June 1996}
\ReportDate{10--06--1996}
\ReportType{Final}
\DatesCovered{1 June 1996 --- 31 May 1999}
\Title{Final Report Title\thanks{We wish to thank all those who helped with
                         this project.  The views and conclusions contained
                         in this document are those of the authors and should
                         not be interpreted as representing anyone else.}}
\ContractNumber{DACA99--99--C--9999}
 %\GrantNumber{}
 %\ProgramElementNumber{}
 %\ProjectNumber{}
 %\TaskNumber{}
 %\WorkUnitNumber{}
\author{Joe A. Uthor and Jane E. Ditor}
\Author{Uthor, Joe, A., and Ditor, Jane, E.}
\PerformingOrg{%
  Our Little Laboratory\\[-1pt]
  Computer Science Department\\[-1pt]
  The Big University\\[-1pt]
  9999 Avenue Street\\[-1pt]
  Anytown, TX  19293--8765}
 %\POReportNumber{}
\SponsoringAgency{%
  U.S. Army Sponsoring Department\\
  P.O. Box 92919, Miltown, AK  97909--6543\\[10pt]
  U.S. Army Monitoring Agency\\
  909 Mission Rd., Somecity, MD  29399--9753}
 %\Acronyms{}
 %\SMReportNumber{}
 %\DistributionStatement{}
\SupplementaryNotes{%
  The views, opinions and/or findings contained in this report are
  those of the authors and should not be construed as an official
  U.S.\ Government position, policy or decision, unless so designated
  by other documentation.}
\Abstract{%
 This sample paper shows one way to use the sf298 package along with
 the article documentclass to generate a paper with the standard form 298
 as the second physical page in the document.

 In addition, the common text of the title, authors and abstract are
 shared and the ``{\tt \string\thanks}'' macro is treated specially so
 that it is active for the titlepage and inactive for the report
 documentation page.}
\SubjectTerms{keywords; associated words; other words}
 %\NumberPages{}
 %\ReportClassification{}
 %\PageClassification{}
 %\AbstractClassification{}
\AbstractLimitation{UU}
\ResponsiblePerson{Mr.\ Joe A. Uthor}
\RPTelephone{(412) 555--9999}
%    \end{macrocode}
% 
% Finally we begin a very simple report, First we print the titlepage
% and the abstract.  Next the report documentation page is written.
% Following these two pages, the table of contents is output, followed by
% the very short report.
% 
%    \begin{macrocode}
\begin{document}
\pagenumbering{roman}

\maketitle\thispagestyle{empty}
\begin{abstract}
 \showabstract
\end{abstract}

\ReportDescription

\tableofcontents
\cleardoublepage
\pagenumbering{arabic}
\setcounter{page}{1}

\section{Introduction}

Blah, blah and blah.

\section{Conclusion}

More blah, blah and blah.

\end{document}
%    \end{macrocode}
% \iffalse
%</sample>
% \fi%
%
% \StopEventually{%
% \section{Acknowledgments}
% 
% I would like to thank Professor Bruce M. Boghosian at the Center for
% Computational Science, Boston University (bruceb@bu.edu) for
% discovering problems with the use of a PostScript background for the
% sf298 form which convinced me to write the entire package in \LaTeX.
%
% \begin{thebibliography}{6}%
% \bibitem{Mull99}%
%   Wihelm M\"uller,
%   \emph{The \regsf{totpages} package},
%   1999/07/13.
% \bibitem{Kopk-Daly93}%
%   Michel Goossens, Framk Mittelbach, and Alexander Samarin,
%   \emph{A Guide to \LaTeXe},
%   1993,  2$^{nd}$ edition (revised 1995).
% \end{thebibliography}}
%
% \section{The {\tt DOCSTRIP} modules}
% The following modules are used in the implementation to direct
% {\tt DOCSCRIPT} in generating the external files:
% \begin{tabular}[t]{ll}
%   driver    & produces a documentation driver file.\\
%   package   & produces the package sf298.sty.\\
%   sample    & produces a sample \LaTeX\ template and test file.
% \end{tabular}
%
% \section{The code}
% \iffalse
%<*package>
% \fi%
% \subsection{Identification}
%
% \changes{v1.1}{15 Feb 2000}{No longer require the graphicx package.}
%
%    \begin{macrocode}
\NeedsTeXFormat{LaTeX2e}
\ProvidesPackage{sf298}[2000/02/22 Standard Form 298]
\typeout{Package: sf298 2000/02/22 v1.1)}
%    \end{macrocode}
%
% \newpage
% \subsection{Load the \regsf{totpages} package, if it exists.}
%
% \begin{macro}{\sf@page}
% \changes{v1.1}{22 Feb 2000}{Added the optional loading of the 
% {\sf totpages} package.  If present, then the total pages block 
% is automatically set.}
%
% Check to see if the \regsf{totpages} package is present.  If it is,
% then load the package and create a local page counter for use in 
% calculating the actual total pages.
%
%    \begin{macrocode}
\IfFileExists{totpages.sty}{%
  \usepackage{totpages}
  \newcounter{sf@page}}{}
%    \end{macrocode}
% \end{macro}
%
% \subsection{Create the user interface}
%
% \changes{v1.1}{22 Feb 2000}{Moved user interface commands and default 
% prior to loading a configuration file}
% 
% The following macros define the external and internal portions of the
% user interface.  These macros are used to fill the various blocks of
% the form.  All but one of the macros have the same pattern: a public
% version which sets the internal value, both based on the block name.
% The exception is the \cmd{\NumberPages} macro that has a starred form
% for special use when the \regsf{totpages} when the page count is 
% calculated automatically (see below).
%
% \begin{macro}{\ReportDate}
% \begin{macro}{\sf@ReportDate}
%    \begin{macrocode}
\newcommand\ReportDate[1]{\renewcommand\sf@ReportDate{#1}}
\newcommand\sf@ReportDate{\number\day--\number\month--\number\year}
%    \end{macrocode}
% \end{macro}
% \end{macro}
%
% \begin{macro}{\ReportType}
% \begin{macro}{\sf@ReportType}
%    \begin{macrocode}
\newcommand\ReportType[1]{\renewcommand\sf@ReportType{#1}}
\newcommand\sf@ReportType{\relax}
%    \end{macrocode}
% \end{macro}
% \end{macro}
%
% \begin{macro}{\DatesCovered}
% \begin{macro}{\sf@DatesCovered}
%    \begin{macrocode}
\newcommand\DatesCovered[1]{\renewcommand\sf@DatesCovered{#1}}
\newcommand\sf@DatesCovered{\relax}
%    \end{macrocode}
% \end{macro}
% \end{macro}
%
% \begin{macro}{\Title}
% \begin{macro}{\sf@Title}
%    \begin{macrocode}
\newcommand\Title[1]{\renewcommand\sf@Title{#1}}
\newcommand\sf@Title{\relax}
%    \end{macrocode}
% \end{macro}
% \end{macro}
%
% \begin{macro}{\ContractNumber}
% \begin{macro}{\sf@ContractNumber}
%    \begin{macrocode}
\newcommand\ContractNumber[1]{\renewcommand\sf@ContractNumber{#1}}
\newcommand\sf@ContractNumber{\relax}
%    \end{macrocode}
% \end{macro}
% \end{macro}
%
% \begin{macro}{\GrantNumber}
% \begin{macro}{\sf@GrantNumber}
%    \begin{macrocode}
\newcommand\GrantNumber[1]{\renewcommand\sf@GrantNumber{#1}}
\newcommand\sf@GrantNumber{\relax}
%    \end{macrocode}
% \end{macro}
% \end{macro}
%
% \begin{macro}{\ProgramElementNumber}
% \begin{macro}{\sf@ProgramElementNumber}
%    \begin{macrocode}
\newcommand\ProgramElementNumber[1]{\renewcommand\sf@ProgramElementNumber{#1}}
\newcommand\sf@ProgramElementNumber{\relax}
%    \end{macrocode}
% \end{macro}
% \end{macro}
%
% \begin{macro}{\ProjectNumber}
% \begin{macro}{\sf@ProjectNumber}
%    \begin{macrocode}
\newcommand\ProjectNumber[1]{\renewcommand\sf@ProjectNumber{#1}}
\newcommand\sf@ProjectNumber{\relax}
%    \end{macrocode}
% \end{macro}
% \end{macro}
%
% \begin{macro}{\TaskNumber}
% \begin{macro}{\sf@TaskNumber}
%    \begin{macrocode}
\newcommand\TaskNumber[1]{\renewcommand\sf@TaskNumber{#1}}
\newcommand\sf@TaskNumber{\relax}
%    \end{macrocode}
% \end{macro}
% \end{macro}
%
% \begin{macro}{\WorkUnitNumber}
% \begin{macro}{\sf@WorkUnitNumber}
%    \begin{macrocode}
\newcommand\WorkUnitNumber[1]{\renewcommand\sf@WorkUnitNumber{#1}}
\newcommand\sf@WorkUnitNumber{\relax}
%    \end{macrocode}
% \end{macro}
% \end{macro}
%
% \begin{macro}{\Author}
% \begin{macro}{\sf@Author}
%    \begin{macrocode}
\newcommand\Author[1]{\renewcommand\sf@Author{#1}}
\newcommand\sf@Author{\relax}
%    \end{macrocode}
% \end{macro}
% \end{macro}
%
% \begin{macro}{\PerformingOrg}
% \begin{macro}{\sf@PerformingOrg}
%    \begin{macrocode}
\newcommand\PerformingOrg[1]{\renewcommand\sf@PerformingOrg{#1}}
\newcommand\sf@PerformingOrg{\relax}
%    \end{macrocode}
% \end{macro}
% \end{macro}
%
% \begin{macro}{\POReportNumber}
% \begin{macro}{\sf@POReportNumber}
%    \begin{macrocode}
\newcommand\POReportNumber[1]{\renewcommand\sf@POReportNumber{#1}}
\newcommand\sf@POReportNumber{\relax}
%    \end{macrocode}
% \end{macro}
% \end{macro}
%
% \begin{macro}{\SponsoringAgency}
% \begin{macro}{\sf@SponsoringAgency}
%    \begin{macrocode}
\newcommand\SponsoringAgency[1]{\renewcommand\sf@SponsoringAgency{#1}}
\newcommand\sf@SponsoringAgency{\relax}
%    \end{macrocode}
% \end{macro}
% \end{macro}
%
% \begin{macro}{\Acronyms}
% \begin{macro}{\sf@Acronyms}
%    \begin{macrocode}
\newcommand\Acronyms[1]{\renewcommand\sf@Acronyms{#1}}
\newcommand\sf@Acronyms{\relax}
%    \end{macrocode}
% \end{macro}
% \end{macro}
%
% \begin{macro}{\SMReportNumber}
% \begin{macro}{\sf@SMReportNumber}
%    \begin{macrocode}
\newcommand\SMReportNumber[1]{\renewcommand\sf@SMReportNumber{#1}}
\newcommand\sf@SMReportNumber{\relax}
%    \end{macrocode}
% \end{macro}
% \end{macro}
%
% \begin{macro}{\DistributionStatement}
% \begin{macro}{\sf@DistributionStatement}
%    \begin{macrocode}
\newcommand\DistributionStatement[1]{%
  \renewcommand\sf@DistributionStatement{#1}}
\newcommand\sf@DistributionStatement{Approval for public release;
                                     distribution is unlimited.}
%    \end{macrocode}
% \end{macro}
% \end{macro}
%
% \begin{macro}{\SupplementaryNotes}
% \begin{macro}{\sf@SupplementaryNotes}
%    \begin{macrocode}
\newcommand\SupplementaryNotes[1]{\renewcommand\sf@SupplementaryNotes{#1}}
\newcommand\sf@SupplementaryNotes{\relax}
%    \end{macrocode}
% \end{macro}
% \end{macro}
%
% \begin{macro}{\Abstract}
% \begin{macro}{\sf@Abstract}
%    \begin{macrocode}
\newcommand\Abstract[1]{\renewcommand\sf@Abstract{#1}}
\newcommand\sf@Abstract{\relax}
%    \end{macrocode}
% \end{macro}
% \end{macro}
%
% \begin{macro}{\SubjectTerms}
% \begin{macro}{\sf@SubjectTerms}
%    \begin{macrocode}
\newcommand\SubjectTerms[1]{\renewcommand\sf@SubjectTerms{#1}}
\newcommand\sf@SubjectTerms{\relax}
%    \end{macrocode}
% \end{macro}
% \end{macro}
%
% \begin{macro}{\ReportClassification}
% \begin{macro}{\sf@ReportClassification}
%    \begin{macrocode}
\newcommand\ReportClassification[1]{\renewcommand\sf@ReportClassification{#1}}
\newcommand\sf@ReportClassification{U}
%    \end{macrocode}
% \end{macro}
% \end{macro}
%
% \begin{macro}{\AbstractClassification}
% \begin{macro}{\sf@AbstractClassification}
%    \begin{macrocode}
\newcommand\AbstractClassification[1]{%
  \renewcommand\sf@AbstractClassification{#1}}
\newcommand\sf@AbstractClassification{U}
%    \end{macrocode}
% \end{macro}
% \end{macro}
%
% \begin{macro}{\PageClassification}
% \begin{macro}{\sf@PageClassification}
%    \begin{macrocode}
\newcommand\PageClassification[1]{\renewcommand\sf@PageClassification{#1}}
\newcommand\sf@PageClassification{U}
%    \end{macrocode}
% \end{macro}
% \end{macro}
%
% \begin{macro}{\AbstractLimitation}
% \begin{macro}{\sf@AbstractLimitation}
%    \begin{macrocode}
\newcommand\AbstractLimitation[1]{\renewcommand\sf@AbstractLimitation{#1}}
\newcommand\sf@AbstractLimitation{UU}
%    \end{macrocode}
% \end{macro}
% \end{macro}
%
% \begin{macro}{\ResponsiblePerson}
% \begin{macro}{\sf@ResponsiblePerson}
%    \begin{macrocode}
\newcommand\ResponsiblePerson[1]{\renewcommand\sf@ResponsiblePerson{#1}}
\newcommand\sf@ResponsiblePerson{\relax}
%    \end{macrocode}
% \end{macro}
% \end{macro}
%
% \begin{macro}{\RPTelephone}
% \begin{macro}{\sf@RPTelephone}
%    \begin{macrocode}
\newcommand\RPTelephone[1]{\renewcommand\sf@RPTelephone{#1}}
\newcommand\sf@RPTelephone{\relax}
%    \end{macrocode}
% \end{macro}
% \end{macro}
%
% \begin{macro}{\NumberPages}
% \begin{macro}{\NumberPages*}
% \changes{v1.1}{29 Feb 2000}{Added ability to offset the automatic
%  number of pages calculation.}
% 
% The \cmd{\NumberPages} macro that has a both a regular form
% (\cmd{\NumberPages}) that is used when setting the number of pages,
% and a starred form (\cmd{\NumberPages*}) that is used to offset
% the automatically calculated number of pages calculated using the
% \regsf{totpages} package.  The value supplied by the starred form 
% is added to account for inserted pages that are not part of the
% \LaTeX\ paper.  For instance, when you are inserting a known number of 
% pages from an alternative source, say for an appendix, into the
% document.  
% 
% The \cmd{\NumberPages} macro looks ahead for an asterisk and if it
% finds one, then it ``eats'' it and passes the argument to the internal
% \cmd{\sf@SNumberPages} macro where it is stored for later use to with the
% total pages calculation.  Otherwise, it sends the argument to
% the \cmd{\sf@NNumberPages} macro where it becomes the actual page number.
%
%    \begin{macrocode}
\newcommand\NumberPages{%
  \@ifnextchar *%
    {\def\@tempa*{\sf@SNumberPages}%
     \@tempa}%
    {\sf@NNumberPages}}
%    \end{macrocode}
% \end{macro}
% \end{macro}
%
% \begin{macro}{\sf@NNumberPages}
% \begin{macro}{\sf@NumberPages}
%    \begin{macrocode}
\newcommand\sf@NNumberPages[1]{\renewcommand\sf@NumberPages{#1}}
\newcommand\sf@NumberPages{\relax}
%    \end{macrocode}
% \end{macro}
% \end{macro}
%
% \begin{macro}{\sf@SNumberPages}
% \begin{macro}{\sf@ExtraPages}
%    \begin{macrocode}
\newcommand\sf@SNumberPages[1]{\renewcommand\sf@ExtraPages{#1}}
\newcommand\sf@ExtraPages{0}
%    \end{macrocode}
% \end{macro}
% \end{macro}
%
% \subsection{Options Processing: Check for a configuration file}
%
% \changes{v1.1}{29 Feb 2000}{Added option to turn on/off the 
% configuration file check.}
% \changes{v1.1}{22 Feb 2000}{Added check for a configuration file.}
%
% First we define a switch that we can set with the optional arguments
% and then declare the optional arguments themselves.
%
%    \begin{macrocode}
\DeclareOption{config}{\def\sf@config{Y}}
\DeclareOption{noconfig}{\def\sf@config{N}}
%    \end{macrocode}
%
% Next we process the optional arguments.  We set the default
% to read the configuration file and process any options to
% allow the user to turn it off.  Then, depending on the 
% current value of the configuration switch, we check for the
% configuration file.
%
%    \begin{macrocode}
\ExecuteOptions{config}
\ProcessOptions
%    \end{macrocode}
%
%    \begin{macrocode}
\if Y\sf@config
  \InputIfFileExists{sf298.cfg}{}{}
\fi
\let\sf@config\relax
%    \end{macrocode}
%
% \subsection{Form placement fine tuning}
%
% \begin{macro}{\LeftShift}
% \begin{macro}{\DownShift}
% 
% These two macros allow the user to set any offset for the form.  This
% allows fine tuning of the form placement as necessary.
%
%    \begin{macrocode}
\newcommand\LeftShift[1]{\renewcommand\sf@LeftShift{#1}}
\newcommand\sf@LeftShift{0}
%    \end{macrocode}
%
%    \begin{macrocode}
\newcommand\DownShift[1]{\renewcommand\sf@DownShift{#1}}
\newcommand\sf@DownShift{0}
%    \end{macrocode}
% \end{macro}
% \end{macro}
%
% \subsection{Font definition}
%
% \begin{macro}{\sf@setfont}
% \changes{v1.1}{15 Feb 2000}{Added titlefonts to support a \LaTeX-based
% blank form (rather than the previous PostScript one).  Added setfont 
% stuff to fix weird things that can be done to upset a fixed format.}
%
% This macro is used to set the baseline and insure that the interword
% spacing has not been changed by evil outside influences when
% typesetting the form titles and entries.
%
%    \begin{macrocode}
\newcommand{\sf@setfont}[2]{%
  \renewcommand{\baselinestretch}{1.0}%
  \setlength{\baselineskip}{#2}%
  \setlength{\normalbaselineskip}{\baselineskip}%
  #1
  \spaceskip=\fontdimen2\font % space between words (\raggedright)
  \xspaceskip=\fontdimen2\font
  \advance\xspaceskip by\fontdimen7\font}
%    \end{macrocode}
% \end{macro}
%
% \begin{macro}{\sf@strut}
% \changes{v1.1}{15 Feb 2000}{Added strut to space the block heading at the
%  exactly proper position in the upper left corner.}
%
% This strut is added at the beginning of each block title to place the
% title text at the desired offset from the upper left corner of the
% block.
%
%    \begin{macrocode}
\newcommand{\sf@strut}{\rule{0pt}{7pt}}
%    \end{macrocode}
% \end{macro}
%
% \begin{macro}{\sf@titlefontA}
% \begin{macro}{\sf@settitlefontA}
% \begin{macro}{\sf@titlefontB}
% \begin{macro}{\sf@settitlefontB}
% \begin{macro}{\sf@titlefontC}
% \begin{macro}{\sf@settitlefontC}
% \begin{macro}{\sf@titlefontD}
% \begin{macro}{\sf@settitlefontD}
% \begin{macro}{\sf@titlefontE}
% \begin{macro}{\sf@settitlefontE}
% \begin{macro}{\sf@titlefontF}
% \begin{macro}{\sf@settitlefontF}
% \changes{v1.1}{22 Feb 2000}{Changed font declarations to use 
% \cmd{\sfdefault} so that they will adjust to the preferred font family.}
%
% The form headings are set with Sans Serif at 14-, 11-, 8-, and 6-point
% sizes and a couple of different shapes and sizes.  Each font is
% actually set by the related \cmd{\settitlefont?} macro which switches 
% to the indicated fixed font, sets the new baseline and protects the 
% text from external settings.  The actual font family used depends on 
% the current setting of \cmd{\sfdefault} \cite[pages 229,236]{Kopk-Daly93}.
%
%    \begin{macrocode}
\DeclareFixedFont{\sf@titlefontA}{OT1}{\sfdefault}{bx}{n}{\@xivpt}
\newcommand{\sf@settitlefontA}{\sf@setfont{\sf@titlefontA}{18pt}}
%    \end{macrocode}
%
%    \begin{macrocode}
\DeclareFixedFont{\sf@titlefontB}{OT1}{\sfdefault}{m}{sl}{\@xipt}
\newcommand{\sf@settitlefontB}{\sf@setfont{\sf@titlefontB}{13.6pt}}
%    \end{macrocode}
%
%    \begin{macrocode}
\DeclareFixedFont{\sf@titlefontC}{OT1}{\sfdefault}{m}{n}{\@vipt}
\newcommand{\sf@settitlefontC}{\sf@setfont{\sf@titlefontC}{7pt}}
%    \end{macrocode}
%
%    \begin{macrocode}
\DeclareFixedFont{\sf@titlefontD}{OT1}{\sfdefault}{bx}{n}{\@vipt}
\newcommand{\sf@settitlefontD}{\sf@setfont{\sf@titlefontD}{7pt}}
%    \end{macrocode}
%
%    \begin{macrocode}
\DeclareFixedFont{\sf@titlefontE}{OT1}{\sfdefault}{bx}{n}{\@viiipt}
\newcommand{\sf@settitlefontE}{\sf@setfont{\sf@titlefontE}{9.5pt}}
%    \end{macrocode}
%
%    \begin{macrocode}
\DeclareFixedFont{\sf@titlefontF}{OT1}{\sfdefault}{m}{sl}{\@viiipt}
\newcommand{\sf@settitlefontF}{\sf@setfont{\sf@titlefontF}{9.5pt}}
%    \end{macrocode}
% \end{macro}
% \end{macro}
% \end{macro}
% \end{macro}
% \end{macro}
% \end{macro}
% \end{macro}
% \end{macro}
% \end{macro}
% \end{macro}
% \end{macro}
% \end{macro}
% \begin{macro}{\sf@entryfont}
% \begin{macro}{\sf@setentryfont}
% \changes{v1.1}{22 Feb 2000}{Changed font declaration to use 
% \cmd{\rmdefault} so that it will adjust to the preferred font family.}
% 
% The form is filled in with Roman 10-point text.  This insures that the
% text will reasonably fit in the form boxes and will not vary with
% the choice of the normal fontsize of the document.  As with the title
% fonts, above, the entry font is actually set by the related
% \cmd{\setentryfont} macro.  The actual font family used depends on
% the current setting of \cmd{\rmdefault} 
% \cite[pages 229,236]{Kopk-Daly93}.
%
%    \begin{macrocode}
\DeclareFixedFont{\sf@entryfont}{OT1}{\rmdefault}{m}{n}{\@xpt}
\newcommand{\sf@setentryfont}{\sf@setfont{\sf@entryfont}{11pt}}
%    \end{macrocode}
% \end{macro}
% \end{macro}
%
% \subsection{Error checking and reporting}
%
% \changes{v1.1}{22 Feb 2000}{Added the error checking and reporting.}
% \begin{macro}{\ifsf@undef}
% \begin{macro}{\sf@undeftrue}
% \begin{macro}{\sf@undeffalse}
% 
% This switch is used by the error reporting at the end of the document
% to generate a general message that there was an undefined required
% field.
% 
%    \begin{macrocode}
\newif\ifsf@undef
\sf@undeffalse
%    \end{macrocode}
% \end{macro}
% \end{macro}
% \end{macro}
%
% \begin{macro}{\sf@undefined}
% 
% The \cmd{\sf@undefined} macro is executed with the optional argument for
% each required field and without the optional argument at the end of
% the document.
%
%    \begin{macrocode}
\newcommand{\sf@undefined}[1][\@empty]{%
  \ifx \@empty#1\relax
    \ifsf@undef
       \typeout{^^J%
                SF298 Warning: There are required fields that are undefined.%
                ^^J}%
     \fi
  \else
    \sf@undeftrue
    \typeout{^^J%
             SF298 Warning: \expandafter\string\csname #1\endcsname\space
             is undefined.%
             ^^J}%
  \fi}
%    \end{macrocode}
% \end{macro}
%
% \begin{macro}{\sf@checkfield}
% 
% This macro does the actual checking for undefined required fields.
% Because of the different sorts of field definitions that could be
% created, it was easier to just fill a box with whatever the user
% defined and check to see if the box has a non-zero width.  You could
% defeat this if you really wanted to, but it should work for reasonable
% field values.
%
%    \begin{macrocode}
\newcommand{\sf@checkfield}[1]{%
  \setbox\@tempboxa\vbox{\csname sf@#1\endcsname}%
  \ifdim \wd\@tempboxa =0pt
    \sf@undefined[#1]
  \fi}
%    \end{macrocode}
% \end{macro}
%
%    \begin{macrocode}
\AtEndDocument{\sf@undefined}
%    \end{macrocode}
%
%
% \subsection{Page layout}
% 
% This is the part of the code that actually sets up the form page and
% draws the form.  After the form is drawn, most of the macros are set
% to \cmd{\relax} in order to recover as much pool space as possible.
%
% \subsubsection{Form setup and cleanup}
% \begin{macro}{\MakeRptDocPage}
% 
% The \cmd{\MakeRptDocPage} command creates the Report Documentation Page
% using the information defined by the associated field macros.  This
% should be used at the beginning of a deliverable report just after
% the cover or title page and counts as the first actual page of the 
% document.
%
%    \begin{macrocode}
\newcommand\MakeRptDocPage{%
%    \end{macrocode}
% 
% If the \regsf{twoside} option is specified in the documentclass, then
% it is printed as a separate page with a blank back.  This page treated
% as page `{\it i}' (if the documentclass \regsf{twoside} option is
% specified, then the back treated as page `{\it ii}'.
%
%    \begin{macrocode}
  \newpage
  \if@twoside\ifodd\c@page\else
    \null
    \thispagestyle{empty}%
    \newpage
  \fi\fi
%    \end{macrocode}
%
% \changes{v1.1}{22 Feb 2000}{Added calculation of total pages (if the
% totpages package is present).  Added check for missing fields.}
%
% Check for missing required fields.  If the field value is missing, then
% print a warning.  In the case of the number of pages, if the 
% \regsf{totpages} package was loaded, then calculate the value, else if 
% the total pages is not given, then give a warning.
%
%    \begin{macrocode}
  \sf@checkfield{ReportDate}
  \sf@checkfield{ReportType}
  \sf@checkfield{DatesCovered}
  \sf@checkfield{Title}
  \sf@checkfield{Author}
  \sf@checkfield{PerformingOrg}
  \sf@checkfield{SponsoringAgency}
  \sf@checkfield{Abstract}
  \sf@checkfield{ReportClassification}
  \sf@checkfield{AbstractClassification}
  \sf@checkfield{PageClassification}
  \sf@checkfield{AbstractLimitation}
%    \end{macrocode}
%
%    \begin{macrocode}
  \edef\@tempa{\sf@NumberPages}%
  \edef\@tempa{\expandafter\@car\@tempa \@nil}%
  \if \relax\@tempa\relax
    \@ifundefined{c@TotPages}%
      {\sf@undefined[NumberPages]}
      {\@ifundefined{r@TotPages}%
         {\renewcommand{\sf@NumberPages}{{\bf ??}}}%
         {\setcounter{sf@page}{\expandafter\@car\r@TotPages\@nil}
          \addtocounter{sf@page}{-\value{TotPages}}
          \addtocounter{sf@page}{\sf@ExtraPages}
          \renewcommand{\sf@NumberPages}{\arabic{sf@page}}}}
  \fi
%    \end{macrocode}
%
% \changes{v1.1}{15 Feb 2000}{Adjusted offset calculations for new
%  blank form.}
% 
% Next we make this page number one of the document and draw the form on
% an $8\frac{1}{2}\times 11$ inch letter-size page and all offsets in
% the actual form layout are in pixels from the lower-left corner of the
% picture box as defined by the \cmd{\sf@drawform} macro. When we close
% off the boxes, we allow the form to expand beyond the zero sized box.
% This keeps \LaTeX\ from issuing warnings about overfull boxes.
%
%    \begin{macrocode}
  \setcounter{page}{1}%
  \thispagestyle{empty}
  \vbox to 0pt{%
    \@tempskipa 34pt
    \advance\@tempskipa\voffset
    \advance\@tempskipa\topmargin
    \advance\@tempskipa\headheight
    \advance\@tempskipa\headsep
    \advance\@tempskipa\baselineskip
    \vskip-\@tempskipa
    \hbox to 0pt{%
      \@tempskipa\oddsidemargin
      \advance\@tempskipa 42pt
      \hskip-\@tempskipa
%    \end{macrocode}
%
%    \begin{macrocode}
      \sf@drawform
%    \end{macrocode}
%
%    \begin{macrocode}
      \hss}%
    \vss}%
%    \end{macrocode}
%
% The page is now done, so, if the documentclass \regsf{twoside} option
% is specified, generate a blank page for its back, which counts as
% page `{\it ii}').
%
%    \begin{macrocode}
  \newpage
  \if@twoside
    \null
    \thispagestyle{empty}%
    \newpage
  \fi
%    \end{macrocode}
%
% \changes{v1.1}{15 Feb 2000}{Added font macros to the list of unused
% definitions.  Also \cmd{\ReportDate} and \cmd{\sf@ReportDate}.}
% Finally, the last step is to set all of the now unused definitions to
% \cmd{\relax}, thereby freeing some space.
%
%    \begin{macrocode}
  \global\let\ReportDate\relax
  \global\let\sf@ReportDate\relax
  \global\let\ReportType\relax
  \global\let\sf@ReportType\relax
  \global\let\DatesCovered\relax
  \global\let\sf@DatesCovered\relax
  \global\let\Title\relax
  \global\let\sf@Title\relax
  \global\let\ContractNumber\relax
  \global\let\sf@ContractNumber\relax
  \global\let\GrantNumber\relax
  \global\let\sf@GrantNumber\relax
  \global\let\ProgramElementNumber\relax
  \global\let\sf@ProgramElementNumber\relax
  \global\let\ProjectNumber\relax
  \global\let\sf@ProjectNumber\relax
  \global\let\TaskNumber\relax
  \global\let\sf@TaskNumber\relax
  \global\let\WorkUnitNumber\relax
  \global\let\sf@WorkUnitNumber\relax
  \global\let\Author\relax
  \global\let\sf@Author\relax
  \global\let\PerformingOrg\relax
  \global\let\sf@PerformingOrg\relax
  \global\let\POReportNumber\relax
  \global\let\sf@POReportNumber\relax
  \global\let\SponsoringAgency\relax
  \global\let\sf@SponsoringAgency\relax
  \global\let\Acronyms\relax
  \global\let\sf@Acronyms\relax
  \global\let\SMReportNumber\relax
  \global\let\sf@SMReportNumber\relax
  \global\let\DistributionStatement\relax
  \global\let\sf@DistributionStatement\relax
  \global\let\SupplementaryNotes\relax
  \global\let\sf@SupplementaryNotes\relax
  \global\let\Abstract\relax
  \global\let\sf@Abstract\relax
  \global\let\SubjectTerms\relax
  \global\let\sf@SubjectTerms\relax
  \global\let\ReportClassification\relax
  \global\let\sf@ReportClassification\relax
  \global\let\AbstractClassification\relax
  \global\let\sf@AbstractClassification\relax
  \global\let\PageClassification\relax
  \global\let\sf@PageClassification\relax
  \global\let\AbstractLimitation\relax
  \global\let\sf@AbstractLimitation\relax
  \global\let\ResponsiblePerson\relax
  \global\let\sf@ResponsiblePerson\relax
  \global\let\RPTelephone\relax
  \global\let\NumberPages\relax
  \global\let\sf@NNumberPages\relax
  \global\let\sf@NumberPages\relax
  \global\let\sf@SNumberPages\relax
  \global\let\sf@ExtraPages\relax
  \global\let\LeftShift\relax
  \global\let\sf@LeftShift\relax
  \global\let\DownShift\relax
  \global\let\sf@DownShift\relax
  \global\let\sf@RPTelephone\relax
  \global\let\sf@setfont
  \global\let\sf@strut\relax
  \global\let\sf@titlefontA\relax
  \global\let\sf@settitlefontA\relax
  \global\let\sf@titlefontB\relax
  \global\let\sf@settitlefontB\relax
  \global\let\sf@titlefontC\relax
  \global\let\sf@settitlefontC\relax
  \global\let\sf@titlefontD\relax
  \global\let\sf@settitlefontD\relax
  \global\let\sf@titlefontE\relax
  \global\let\sf@settitlefontE\relax
  \global\let\sf@titlefontF\relax
  \global\let\sf@settitlefontF\relax
  \global\let\sf@entryfont\relax
  \global\let\sf@setentryfont\relax
  \global\let\sf@checkfield\relax
  \global\let\sf@parbox\relax
  \global\let\sf@drawform\relax
  \global\let\MakeRptDocPage\relax}
%    \end{macrocode}
% \end{macro}
%
% \subsubsection{Form layout}
% \begin{macro}{\sf@parbox}
% \changes{v1.1}{15 Feb 2000}{Added.}
%
% This macro is used in the following \cmd{\sf@drawform} to contain fields that
% are to be centered in a block, but expand upward when overfull.
%    \begin{macrocode}
\def\sf@parbox(#1,#2)#3{%
  \makebox(0,0)[bl]{%
    \vbox to #1\unitlength{%
      \vss
      \noindent
      \parbox{#2\unitlength}{\raggedright#3}%
    \vfil}}}
%    \end{macrocode}
% \end{macro}
%
% \begin{macro}{\sf@drawform}
% \changes{v1.1}{15 Feb 2000}{Changed from millimeter to big point spacing.}
%
% First draw the boxes of the form as a large thick-walled box with
% a series of thin horizontal and vertical lines.
%
%    \begin{macrocode}
\newcommand\sf@drawform{%
  \sf@setentryfont
  \setlength{\unitlength}{1bp}
  \begin{picture}(539,707)(\sf@LeftShift,\sf@DownShift)
    % Draw blank form
    \linethickness{1bp}%
    \put(  0,  0){\framebox(539,707){}}
    % Horizontal lines
    \linethickness{0.5bp}%
    \put(  0,677){\line(1,0){539}}
    \put(  0,635){\line(1,0){539}}
    \put(  0,611){\line(1,0){539}}
    \put(324,581){\line(1,0){215}}
    \put(324,551){\line(1,0){215}}
    \put(  0,521){\line(1,0){539}}
    \put(324,491){\line(1,0){215}}
    \put(324,461){\line(1,0){215}}
    \put(  0,431){\line(1,0){539}}
    \put(  0,372){\line(1,0){539}}
    \put(359,336){\line(1,0){180}}
    \put(  0,300){\line(1,0){539}}
    \put(  0,252){\line(1,0){539}}
    \put(  0,215){\line(1,0){539}}
    \put(  0, 95){\line(1,0){539}}
    \put(  0, 48){\line(1,0){539}}
    \put(  0, 36){\line(1,0){180}}
    \put(324, 24){\line(1,0){215}}
    % Vertical lines
    \put(359,707){\line(0,-1){ 30}}
    \put(144,635){\line(0,-1){ 24}}
    \put(359,635){\line(0,-1){ 24}}
    \put(324,611){\line(0,-1){180}}
    \put(359,431){\line(0,-1){131}}
    \put(180, 48){\line(0,-1){ 48}}
    \put(266, 48){\line(0,-1){ 48}}
    \put(324, 48){\line(0,-1){ 48}}
    \put( 60, 36){\line(0,-1){ 36}}
    \put(120, 36){\line(0,-1){ 36}}
%    \end{macrocode}
%
% Next, fill in the individual box titles.  Note the use of the 
% \cmd{\sf@strut} to set the block titles a little down and to 
% the right of the absolute corner of the box.
%
%    \begin{macrocode}
    % Title text
    \put(  0,677){\makebox(359, 30){%
      \sf@settitlefontA REPORT DOCUMENTATION PAGE}}
    \put(359,677){\makebox(180, 30){%
      \sf@settitlefontB\shortstack[c]{%
        Form Approved\\[-2pt]%
        OMB No. 0704--0188}}}
    \put(  3,635){\makebox(533, 42){%
      \sf@settitlefontC\parbox{533bp}{\raggedright
      The public reporting burden for this collection of information
      is estimated to average 1 hour per response, including the
      time for reviewing instructions, searching existing data
      sources, gathering and maintaining the data needed, and
      completing and reviewing the collection of information.
      Send comments regarding this burden estimate or any other
      aspect of this collection of information, including
      suggestions for reducing this burden to Department of
      Defense, Washington Headquarters Services, Directorate for
      Information Operations and Reports (0704--0188), 1215
      Jefferson Davis Highway, Suite 1204, Arlington, VA
      22202--4302.  Respondents should be aware that
      notwithstanding any other provision of law, no person shall
      be subject to any penalty for failing to comply with a
      collection of information if it does not display a currently
      valid OMB control number.
      {\sf@settitlefontD
       PLEASE DO NOT RETURN YOUR FORM TO THE ABOVE ADDRESS.}}}}
    \put(  0,611){\makebox(144, 24)[tl]{%
      \sf@settitlefontE\sf@strut 1.\ REPORT DATE
      \sf@settitlefontF (DD--MM--YYYY)}}
    \put(144,611){\makebox(215, 24)[tl]{%
      \sf@settitlefontE\sf@strut 2.\ REPORT TYPE}}
    \put(359,611){\makebox(180, 24)[tl]{%
      \sf@settitlefontE\sf@strut 3.\ DATES COVERED
      \sf@settitlefontF (From --- To)}}
    \put(  0,521){\makebox(324,90)[tl]{%
      \sf@settitlefontE\sf@strut 4.\ TITLE AND SUBTITLE}}
    \put(324,581){\makebox(215, 30)[tl]{%
      \sf@settitlefontE\sf@strut 5a.\ CONTRACT NUMBER}}
    \put(324,551){\makebox(215, 30)[tl]{%
      \sf@settitlefontE\sf@strut 5b.\ GRANT NUMBER}}
    \put(324,521){\makebox(215, 30)[tl]{%
      \sf@settitlefontE\sf@strut 5c.\ PROGRAM ELEMENT NUMBER}}
    \put(324,491){\makebox(215, 30)[tl]{%
      \sf@settitlefontE\sf@strut 5d.\ PROJECT NUMBER}}
    \put(324,461){\makebox(215, 30)[tl]{%
      \sf@settitlefontE\sf@strut 5e.\ TASK NUMBER}}
    \put(324,431){\makebox(215, 30)[tl]{%
      \sf@settitlefontE\sf@strut 5f.\ WORK UNIT NUMBER}}
    \put(  0,431){\makebox(324, 90)[tl]{%
      \sf@settitlefontE\sf@strut 6.\ AUTHOR(S)}}
    \put(  0,372){\makebox(359, 59)[tl]{%
      \sf@settitlefontE\sf@strut
      7.\ PERFORMING ORGANIZATION NAME(S) AND ADDRESS(ES)}}
    \put(359,372){\makebox(180, 59)[tl]{%
      \sf@settitlefontE\shortstack[l]{%
        \sf@strut 8.\ PERFORMING ORGANIZATION REPORT\\[-2pt]
        \sf@strut\phantom{8.\ }NUMBER}}}
    \put(  0,300){\makebox(359, 72)[tl]{%
      \sf@settitlefontE\sf@strut
      9.\ SPONSORING / MONITORING AGENCY NAME(S) AND ADDRESS(ES)}}
    \put(359,336){\makebox(180, 36)[tl]{%
      \sf@settitlefontE\sf@strut 10.\ SPONSOR/MONITOR'S ACRONYM(S)}}
    \put(359,300){\makebox(180, 36)[tl]{%
      \sf@settitlefontE\shortstack[l]{%
        \sf@strut 11.\ SPONSOR/MONITOR'S REPORT\\[-3pt]
        \sf@strut\phantom{11.\ }NUMBER(S)}}}
    \put(  0,252){\makebox(539, 48)[tl]{%
      \sf@settitlefontE\sf@strut
      12.\ DISTRIBUTION / AVAILABILITY STATEMENT}}
    \put(  0,215){\makebox(539, 37)[tl]{%
      \sf@settitlefontE\sf@strut 13.\ SUPPLEMENTARY NOTES}}
    \put(  0,95){\makebox(539,120)[tl]{%
      \sf@settitlefontE\sf@strut 14.\ ABSTRACT}}
    \put(  0, 48){\makebox(539, 46)[tl]{%
      \sf@settitlefontE\sf@strut 15.\ SUBJECT TERMS}}
    \put(  0, 36){\makebox(180, 12)[tl]{%
      \sf@settitlefontE\sf@strut 16.\ SECURITY CLASSIFICATION OF:}}
    \put(  0,  0){\makebox( 60, 36)[tl]{%
      \sf@settitlefontE\sf@strut a.\ REPORT}}
    \put( 60,  0){\makebox(60, 36)[tl]{%
      \sf@settitlefontE\sf@strut b.\ ABSTRACT}}
    \put(120,  0){\makebox(60, 36)[tl]{%
      \sf@settitlefontE\sf@strut c.\ THIS PAGE}}
    \put(180,  0){\makebox( 86, 48)[tl]{%
      \sf@settitlefontE\shortstack[l]{%
        \sf@strut 17.\ LIMITATION OF\\[-2pt]
        \sf@strut\phantom{17.\ }ABSTRACT}}}
    \put(266,  0){\makebox( 68, 48)[tl]{%
      \sf@settitlefontE\shortstack[l]{%
        \sf@strut 18.\ NUMBER\\[-2pt]
        \sf@strut\phantom{18.\ }OF\\[-2pt]
        \sf@strut\phantom{18.\ }PAGES}}}
    \put(324, 24){\makebox(215, 24)[tl]{%
      \sf@settitlefontE\sf@strut 19a.\ NAME OF RESPONSIBLE PERSON}}
    \put(324,  0){\makebox(215, 24)[tl]{%
      \sf@settitlefontE\sf@strut 19b.\ TELEPHONE NUMBER
        \sf@settitlefontF (include area code)}}
    \put(415,-17){\makebox(123, 17)[bl]{%
      \shortstack[l]{%
        {\sf@settitlefontE Standard Form 298 (Rev.\ 8--98)}\\[-2pt]%
        {\sf@settitlefontD Prescribed by ANSI Std.\ Z39.18}}}}
%    \end{macrocode}
%
% Now, we add the user's entries to the form.  For undefined entries,
% we just add null text, the required fields are checked above and 
% warning issued if any of them are empty.
%
%    \begin{macrocode}
    % Fill in form
    \put(  4,614){\makebox(0,0)[bl]{\sf@ReportDate}}
    \put(148,614){\makebox(0,0)[bl]{\sf@ReportType}}
    \put(363,614){\makebox(0,0)[bl]{\sf@DatesCovered}}
    \put(  4,523){\sf@parbox( 80,324){\sf@Title}}
    \put(328,585){\makebox(0,0)[bl]{\sf@ContractNumber}}
    \put(328,555){\makebox(0,0)[bl]{\sf@GrantNumber}}
    \put(328,525){\makebox(0,0)[bl]{\sf@ProgramElementNumber}}
    \put(328,495){\makebox(0,0)[bl]{\sf@ProjectNumber}}
    \put(328,465){\makebox(0,0)[bl]{\sf@TaskNumber}}
    \put(328,435){\makebox(0,0)[bl]{\sf@WorkUnitNumber}}
    \put(  4,434){\sf@parbox( 80,321){\sf@Author}}
    \put(  4,397){\makebox(0,0)[l]{%
      \parbox{355bp}{\raggedright\sf@PerformingOrg}}}
    \put(363,372){\sf@parbox( 46,176){\sf@POReportNumber}}
    \put(  4,331){\makebox(0,0)[l]{%
      \parbox{355bp}{\raggedright \sf@SponsoringAgency}}}
    \put(363,349){\makebox(0,0)[l]{%
      \parbox{176bp}{\raggedright\sf@Acronyms}}}
    \put(363,304){\makebox(0,0)[bl]{%
      \vbox to 26bp{%
        \vss
        \noindent
        \parbox{176bp}{\raggedright\sf@SMReportNumber}}}}
    \put(  4,255){\sf@parbox( 38,532){\sf@DistributionStatement}}
    \put(  4,217){\sf@parbox( 25,532){\sf@SupplementaryNotes}}
    \put(  4, 95){\sf@parbox(110,533){\sf@Abstract}}
    \put(  4, 51){\sf@parbox( 36,533){\sf@SubjectTerms}}
    \put( 30, 13){\makebox(0,0){\sf@ReportClassification}}
    \put( 90, 13){\makebox(0,0){\sf@AbstractClassification}}
    \put(150, 13){\makebox(0,0){\sf@PageClassification}}
    \put(223, 13){\makebox(0,0){\sf@AbstractLimitation}}
    \put(295, 13){\makebox(0,0){\large\sf@NumberPages}}
    \put(328, 33){\makebox(0,0)[l]{\sf@ResponsiblePerson}}
    \put(328,  9){\makebox(0,0)[l]{\sf@RPTelephone}}
  \end{picture}}
%    \end{macrocode}
% \end{macro}
%
% \iffalse
%</package>
% \fi%
%
% \Finale
%

